\documentclass[12pt]{beamer}
\usepackage{color}
\usepackage{xcolor}
%\usepackage{latexsym}
%\usepackage{unicode-math}
\usepackage{amsmath}
\usepackage[labelfont=bf]{caption}
\usepackage{graphicx}  %package graphic
%\usepackage{siunitx}
\usepackage{tikz}
%\usepackage{algorithmicx}
%\usepackage[noend]{algpseudocode}
\usetikzlibrary{automata, positioning, arrows}
\usetheme{Boadilla}   
\usepackage{xeCJK}
%\usepackage{array}
%\usepackage{tabularx}
%\usepackage{mathtools}
%\usepackage{listings}
%\usepackage{textcomp}
%\usepackage[T1]{fontenc}
%\usepackage{lmodern}
\usepackage{stmaryrd}
\usepackage{adjustbox}

\setCJKmainfont{微軟正黑體} 

\setbeamerfont{title}{size=\Large,series=\bfseries}  % title size

\setbeamerfont{frametitle}{size=\large,series=\bfseries}  % frametitle size, also can size*=<pt>

% Item include picture
\setbeamertemplate{itemize item}   % First Level item
{\includegraphics[height=0.33cm]{../Figures/golden-earth-on-white}}

\setbeamertemplate{itemize subitem} % Second level item
{\includegraphics[height=0.31cm]{../Figures/golden-sun-on-white}}

\setbeamertemplate{itemize subsubitem} % Third Level item
{\includegraphics[height=0.27cm]{../Figures/golden-paw-on-white}}

\definecolor{darkgold}{rgb}{0.765 0.64 0.0} % for highlighted text in black-and-white slides

%\newtheorem{problem}{Problem}
\mode<presentation>{\newtheorem{algorithm}{Algorithm}}
\mode<article>{\newenvironment{algorithm}{}{}}
%\newtheorem{solution}{Solution}

%\renewcommand{\highlightb}{\highlightg}

%\newlength{\subtextwidth}
%\setlength{\subtextwidth}{11cm}


\title{Antichains: Alternative Algorithms for LTL Satisfiability and Model-Checking}
%\subtitle{(Based on [Manber 1989])}
\author{林宏陽}
%\institute[IM.NTU]{Department of Information Management\\ National Taiwan University}
%\date[Algorithms 2019]{\null}
\mode<presentation>{\date[SVVRL]{\null}}
\mode<article>{\date{\today}}

\begin{document}
\begin{frame}
	\maketitle
\end{frame}

\begin{frame}{Introduction}
	\begin{itemize}
		\item A model for an LTL formula over a set $P$ of propositions is an infinite word $\mathnormal{w}$ over the alphabet $\Sigma = 2^{P}$.
		\item An LTL formula $\phi$ defines a set of words $\llbracket \phi \rrbracket = \{\mathnormal{w} \in \Sigma^{\omega} \mid \mathnormal{w} \models \phi\}$.
		\item The satisfiability problem asks, given an LTL formula $\phi$, if $\llbracket \phi \rrbracket$ is empty.
		\item The model-checking problem asks, given an omega-regular language $\mathcal{L}$(e.g., the set
of all computations of a reactive system) and a LTL formula $\phi$, if $\mathcal{L} \subseteq \llbracket \phi \rrbracket$
		\item Given a LTL formula $\phi$, we can construct a NBW $\mathcal{A}_{\phi}$ such that $\text{L}_{\text{b}}(\mathcal{A}_{\phi}) = \llbracket \phi \rrbracket$
		\item This reduces the satisfiability and model-checking problems to automata-theoretic questions.
	\end{itemize}
\end{frame}

\begin{frame}{Introduction(cont'd)}
	\begin{itemize}
		\item Implemented in explicit way(such as SPIN) : worst-case exponential.
		\item Implemented in symbolic way(such as NuSMV) : it scales better.
		\item Efficient algorithms to reason on large LTL formulas are highly desirable.
		\item A new efficient way to analyze sABW (the number of states and symbolic transitions are linear in the size of the LTL formula).
		\item This approach avoids the explicit translation to NBW.
	\end{itemize}
\end{frame}

\begin{frame}{Linear Temporal Logic}
	\begin{itemize}
		\item $P$ is a finite set of propositions
		\item a Kripke structure over
$P$ is a tuple $\mathcal{K} = \langle Q, q_{\iota}, \rightarrow_{\mathcal{K}}, \mathcal{L} \rangle$
		\begin{itemize}
			\item $Q$ is a finite set of states	
			\item $q_{\iota} \in Q$ is the initial state
			\item $\rightarrow_{\mathcal{K}} \subseteq Q \times Q$ is a transition relation
			\item $\mathcal{L} : Q \to 2^{P}$ is a labeling function
		\end{itemize}
		\item A run of $\mathcal{K}$ is an infinite sequence $\rho = q_{0}q_{1}\cdots$ such that $q_{0} = q_{\iota}$ and for all $i \geq 0$, $(q_{i}, q_{i+1}) \in \rightarrow_{\mathcal{K}}$
		\item Let $\mathcal{L}(\rho) = \mathcal{L}(q_{0})\mathcal{L}(q_{1})\cdots$ and define the language of K as $\text{L}(\mathcal{K}) =
\{\mathcal{L}(\rho) \mid \rho \text{ is a run of } \mathcal{K}\}$
	\end{itemize}
\end{frame}

\begin{frame}{Linear Temporal Logic(cont'd)}
	\begin{itemize}
		\item The LTL formulas over $P$ are defined by $\phi ::= p \mid \neg\phi \mid \phi \lor \phi \mid \mathbin{\mathcal{X}}\phi \mid \phi \mathbin{\mathcal{U}} \phi$ where $p \in P$.
		\item Given an infinite word $\mathnormal{w} = \sigma_{0}\sigma_{1}\cdots \in \Sigma^{\omega}$ where $\Sigma = 2^{P}$ , and an LTL formula $\phi$ over $P$, we say that $\mathnormal{w}$ satisfies $\phi$ (written $\mathnormal{w} \models \phi$) if and only if (recursively):
		\begin{itemize}
			\item $\phi \equiv p$ and $p \in \sigma_{0}$,
			\item $\phi \equiv \neg\phi_{1}$ and $\mathnormal{w} \not\models \phi_{1}$,
			\item $\phi \equiv \phi_{1} \lor \phi_{2}$ and $\mathnormal{w} \models \phi_{1}$ or $\mathnormal{w} \models \phi_{2}$,
			\item $\phi \equiv \mathbin{\mathcal{X}}\phi_{1}$ and $\sigma_{1}\sigma_{2}\cdots \models \phi_{1}$, or
			\item $\phi \equiv \phi_{1}\mathbin{\mathcal{U}}\phi_{2}$ and for some $k \in N$, $\sigma_{k}\sigma_{k+1}\cdots \models \phi_{2}$ and for all $i$, $0 \leq i < k$,
$\sigma_{i}\sigma_{i+1}\cdots \models \phi_{1}$.
		\end{itemize}
		\item $true$, $false$ and $\phi_{1} \land \phi_{2}$ can be derived from the definition.
		\item let $\Diamond\phi = \text{true} \mathbin{\mathcal{U}}\phi$, $\Box\phi =
\neg\Diamond\neg\phi$, and $\phi_{1}\mathbin{\mathcal{R}}\phi_{2} = \neg(\neg\phi_{1}\mathbin{\mathcal{U}}\neg\phi_{2})$.
	\end{itemize}
\end{frame}

\begin{frame}{Linear Temporal Logic(cont'd)}
	\begin{itemize}
		\item The satisfiability problem : whether $\llbracket \phi \rrbracket = \emptyset$.
		
		\item The model-checking problem : whether $\mathcal{K} \models \phi$
		\begin{itemize}
			\item we say that $\mathcal{K}$ satisfies $\phi$ ($\mathcal{K} \models \phi$) if and only if $\text{L}(\mathcal{K}) \subseteq \llbracket \phi \rrbracket$
			\item that is for all runs $\rho$ of $\mathcal{K}$, we have $\mathcal{L}(\rho) \models \phi$
		\end{itemize}
		\item Both problems are PSPACE-COMPLETE.
	\end{itemize}
\end{frame}

\begin{frame}{sABW}
	\begin{itemize}
		\item Given a finite set $Q$, let $\text{Lit}(Q) = Q \cup \{\neg q \mid q \in Q\}$ be the set of literals over $Q$.
		\item $\mathcal{B}^{+}(Q)$ be the set of positive boolean formulas over $Q$ (formulas built from elements in $Q\cup\{true, false\}$ using $\land$ and $\lor$)
		\item Given $R \subseteq Q$ and $\varphi \in \mathcal{B}^{+}(Q)$, $R \models \varphi$ if and only if we assign $true$ to the
elements of $R$ and $false$ to the elements of $Q \setminus R$ satisfies $\varphi$.
	\end{itemize}
\end{frame}

\begin{frame}{sABW(cont'd)}
	\begin{itemize}
		\item A symbolic alternating Büchi automaton over the set of propositions $P$ is a tuple $\mathcal{A} = \langle \text{Loc}, I, \Sigma, \delta, \alpha \rangle$ where
		\begin{itemize}
			\item Loc is a finite set of states (or locations)
			\item $I \in \mathcal{B}^{+}(\text{Loc})$.
				\begin{itemize}
					\item A set $s \subseteq \text{Loc}$ is initial if $s \models I$.
				\end{itemize}
			\item $\Sigma = 2^{P}$ is the alphabet.
			\item $\delta : \text{Loc} \to \mathcal{B}^{+}(\text{Lit}(P)\cup\text{Loc})$ is the transition function.
			\item $\alpha \subseteq \text{Loc}$ is the set of accepting states.
		\end{itemize}
	\end{itemize}
\end{frame}


\begin{frame}{sABW(cont'd)}
	\begin{itemize}
		\item A run of $\mathcal{A}$ on an infinite word $\mathnormal{w} = \sigma_{0}\sigma_{1}\cdots \in \Sigma^{\omega}$ is a DAG $T_{\mathnormal{w}} = \langle V, V_{\iota}, \to \rangle$ where:
		\begin{itemize}
			\item $V \subseteq \text{Loc} \times N$. A node $(\ell, \mathnormal{i})$ represents the location $\ell$ after the first $\mathnormal{i}$ letters of $\mathnormal{w}$ have been read by $\mathcal{A}$. Nodes of the form $(\ell, \mathnormal{i})$ with 	$\ell \in \alpha$ are called $\alpha$-nodes;
			\item $V_{\iota} \subseteq \text{Loc} \times \{0\}$ is such that $V_{\iota} \subseteq V$ and $\{\ell \mid (\ell, 0) \in V_{\iota}\} \models I$;
			\item $\to \subseteq V \times V$ is such that for all $(\ell, \mathnormal{i}) \in V$
				\begin{itemize}
					\item if $(\ell, \mathnormal{i}) \to (\ell', \mathnormal{i}')$ then $\mathnormal{i}' = \mathnormal{i} + 1$
					\item $\sigma_{i} \cup \{\ell' \mid (\ell, \mathnormal{i}) \to (\ell', \mathnormal{i} + 1)\} \models \delta(\ell)$
				\end{itemize}
		\end{itemize}
		\item A run $T_{\mathnormal{w}}$ on an infinite word $\mathnormal{w}$ is accepting if \textcolor{red}{all} its infinite paths visit $\alpha$-nodes infinitely often.
		\item An infinite word $\mathnormal{w} \in \Sigma^{\omega}$ is accepted by $\mathcal{A}$ if there \textcolor{red}{exists} an accepting run on it.
	\end{itemize}
\end{frame}

\newcommand{\ABWExample}[1]{
\tikzset{
->, % makes the edges directed
>=stealth, % makes the arrow heads bold
every edge/.style={draw, thick, black},
node distance=0.8cm and 2.3cm, % specifies the minimum distance between two nodes. Change if necessary.
every state/.style={thick, fill=gray!10}, % sets the properties for each ’state’ node
initial text=$ $, % sets the text that appears on the start arrow
}
	\begin{figure}[ht] % ’ht’ tells LaTeX to place the figure ’here’ or at the top of the page
	\centering % centers the figure
	\begin{adjustbox}{max width=#1\textwidth, max totalheight=#1\textheight}
		\begin{tikzpicture}
			\node[draw=none] (l) {};
			\node[state, accepting, above right = 0.6cm and 1.5cm of l] (l4) {$\ell_{4}$};
			\node[state, right = of l4] (l3) {$\ell_{3}$};
			\node[state, below right =0.6cm and 1.5cm of l] (l2) {$\ell_{2}$};
			\node[state, accepting, right = of l2] (l1) {$\ell_{1}$};
			\node[draw=none, below right =0.4cm and 1cm of l3] (ll) {};
			\draw (l) edge[bend right = 25] node{} (l4);
			\draw (l) edge[bend left = 25] node{} (l2);
			\draw (l4) edge[loop above] node{$p$} (l4);
			\draw (l3) edge[loop above] node{true} (l3);
			\draw (l2) edge[loop below] node{true} (l2);
			\draw (l1) edge[loop below] node{$\neg r$} (l1);
			\draw (l4) edge node[above]{true} (l3);
			\draw (l4) edge[loop, out=0, in=-50, looseness=6] (l4);
			\draw (l2) edge node[above]{$\neg p \land \neg r$} (l1);
			\draw (l3) edge node[above]{$p$} (ll);
		\end{tikzpicture}
	\end{adjustbox}
		\caption{Alternating automaton for $\varphi \equiv \neg(\Box\Diamond p \to \Box(\neg p \to \Diamond r))$}
	\end{figure}
}


\begin{frame}{example}
	\ABWExample{0.45}
	\begin{itemize}
		\item $\ell_{4}$ and $\ell_{3}$ check that $\phi_{1} \equiv \Box\Diamond p$ holds.
		\item $\ell_{2}$ and $\ell_{1}$ check that $\phi_{2} \equiv \Diamond(\neg p \land \Box\neg r)$ holds.
		\item $\delta(\ell_{4}) = (p \land \ell_{4}) \lor (\text{true} \land \ell_{3} \land \ell_{4})$
		\item $\delta(\ell_{3}) = (\text{true}\land \ell_{3})\lor (p \land \text{true})$
	\end{itemize}
\end{frame}

\begin{frame}{sNBW}
	\begin{itemize}
		\item A nondeterministic Büchi automaton is an sABW $\mathcal{A} = \langle \text{Loc}, I, \Sigma, \delta, \alpha \rangle$ such that
		\begin{itemize}
			\item $I$ is a disjunction of locations.
			\item for all $\ell \in \text{Loc}$, $\delta(\ell)$ is a disjunction of $\varphi \land \ell'$ where $\varphi \in \mathcal{B}^{+}(\text{Lit}(P))$ and $ \ell' \in \text{Loc}$.
		\end{itemize}
		\item Define the reverse automaton of $\mathcal{A}$ as $\mathcal{A}^{-1} = \langle \text{Loc}, I, \Sigma, \delta^{-1}, \alpha \rangle$ where $\delta^{-1} = \{(\ell, \sigma, \ell') \mid (\ell', \sigma, \ell) \in \delta\}$.
	\end{itemize}
\end{frame}

\begin{frame}{Miyano-Hayashi construction}
	\begin{itemize}
		\item Transform an sABW into a sNBW.
		\item The state of sNBW is the form $\langle s, o \rangle$.
		\item $s$ is a set of states that maintains a whole level of a guessed run DAG.
		\item $o$ is a set of states that ``owe'' a visit to an accepting state.
		\item  The Büchi condition asks that $o$ gets empty infinitely often in order to ensure that \textcolor{red}{every} path of the run DAG visits accepting states infinitely often.
	\end{itemize}
\end{frame}

\begin{frame}{Miyano-Hayashi construction (cont'd)}
	
		Given an sABW $\mathcal{A} = \langle \text{Loc}, I, \Sigma, \delta, \alpha \rangle$ over $P$, let MH$(\mathcal{A}) = \langle Q, I^{\text{MH}}, \Sigma, \delta^{\text{MH}}, \alpha^{\text{MH}} \rangle$ be a sNBW where:
		\begin{itemize}
			\item $Q = 2^{\text{Loc}} \times 2^{\text{Loc}}$
			\item $I^{\text{MH}}$ is the disjunction of all the pairs $\langle s$, $\emptyset \rangle$ such that $s \models I$
		
		\item $\delta^{\text{MH}}$ is defined for all $\langle s, o \rangle \in Q$:
		\begin{itemize}
			\item if $o \neq \emptyset$, then $\delta^{\text{MH}}(\langle s, o \rangle)$ is the disjunction of all the formulas $\varphi \land \langle s', o' \setminus \alpha \rangle$ with $\varphi \in \mathcal{B}^{+}(\text{Lit}(P))$ such that
			\begin{itemize}
				\item $o' \subseteq s'$
				\item $\forall \ell \in s \cdot \forall \sigma \subseteq P$ : if $\sigma \models \varphi$ then $\sigma \cup s' \models \delta(\ell)$
				\item $\forall \ell \in o \cdot \forall \sigma \subseteq P$ : if $\sigma \models \varphi$ then $\sigma \cup o' \models \delta(\ell)$
			\end{itemize}
			
			\item if $o = \emptyset$, then $\delta^{\text{MH}}(\langle s, o \rangle)$ is the disjunction of all the formulas $\varphi \land \langle s', s' \setminus \alpha \rangle$ with $\varphi \in \mathcal{B}^{+}(\text{Lit}(P))$ such that
			\begin{itemize}
				\item $\forall \ell \in s \cdot \forall \sigma \subseteq P$ : if $\sigma \models \varphi$ then $\sigma \cup s' \models \delta(\ell)$
			\end{itemize}
		\end{itemize}
		\item $\alpha^{\text{MH}} = 2^{\text{Loc}} \times \{\emptyset\}$
		\end{itemize}
	\begin{theorem}
		For all sABW $\mathcal{A}$, we have $\text{L}_{\text{b}}(\text{MH}(\mathcal{A})) = \text{L}_{\text{b}}(\mathcal{A})$.
	\end{theorem}
\end{frame}

\begin{frame}{Recall ABW}
	\ABWExample{1}
\end{frame}

\begin{frame}{After Miyano-Hayashi construction}
\tikzset{
%->, % makes the edges directed
>=stealth, % makes the arrow heads bold
every edge/.style={draw, thick, black},
node distance=0.8cm and 2.3cm, % specifies the minimum distance between two nodes. Change if necessary.
every state/.style={thick, fill=gray!10}, % sets the properties for each ’state’ node
initial text=$ $, % sets the text that appears on the start arrow
}
	\begin{figure}[t] % ’ht’ tells LaTeX to place the figure ’here’ or at the top of the page
	\centering % centers the figure
	\begin{adjustbox}{max width=.95\textwidth, max totalheight=.95\textheight}
		\begin{tikzpicture}
			\node[state] (a) {$\langle \{\ell_{2}, \ell_{4}\}, \{\ell_{2}\}\rangle$};
			\node[state, initial, accepting, left = of a] (1) {$\langle \{\ell_{2}, \ell_{4}\}, \emptyset$};
			\node[state, above right = and 4.5cm of a] (b) {$\langle \{\ell_{2}, \ell_{3},\ell_{4}\}, \{\ell_{2}\}\rangle$};
			\node[state, above left = of b] (2) {$\langle \{\ell_{2}, \ell_{3}, \ell_{4}\}, \{\ell_{2}, \ell_{3}\}\rangle$};
			\node[state, above right = of b] (3) {$\langle \{\ell_{2}, \ell_{3}, \ell_{4}\}, \{\ell_{3}\}\rangle$};
			\node[state, accepting, below right = of a] (c) {$\langle \{\ell_{1}, \ell_{3},\ell_{4}\}, \emptyset\rangle$};
			\node[state, accepting, right = of c] (d) {$\langle \{\ell_{1}, \ell_{4}\}, \emptyset\rangle$};
			\node[state, right = of d] (e) {$\langle \{\ell_{1}, \ell_{3},\ell_{4}\}, \{\ell_{3}\}\rangle$};


			\draw (3.north west) edge[->, bend right = 100] node[left]{$\{p\}, \{p, r\}$} (1);
			\draw (1) edge[->] node[above]{$\{p\}, \{p, r\}$} (a);
			\draw (1) edge[->, bend left = 20] node[above]{$\emptyset, \{p\}, \{r\}, \{p, r\}$} (2);
			\draw (1) edge[->, bend right = 40] node[below]{$\emptyset$} (e);
			\draw (2) edge[loop right] node{$\emptyset, \{p\}, \{r\}, \{p, r\}$} (2);
			\draw (2.south) edge[->, bend right = 15] node[above]{$\{p\}, \{p, r\}$} (b);
			\draw (2.north east) edge[->, bend left = 10] node[above]{$\emptyset$} (3);
			\draw (3) edge[loop above] node{$\emptyset, \{p\}, \{r\}, \{p, r\}$} (3);
			\draw (3) edge[->, bend left = 20] node[right]{$\emptyset$} (e);
			\draw (a) edge[loop above] node{$\{p\}, \{p, r\}$} (a);
			\draw (a) edge[->, bend left = 15] node[above = 0.5cm]{$\emptyset, \{p\}, \{r\}, \{p, r\}$} (b);
			\draw (b) edge[->, bend left = 15] node[below = 0.3cm]{$\{p\}, \{p, r\}$} (a);
			\draw (b) edge[loop right] node{$\emptyset, \{p\}, \{r\}, \{p, r\}$} (b);
			\draw (b) edge[->] node[right]{$\emptyset$} (c);
			\draw (a) edge[->] node[above]{$\emptyset$} (c);
			\draw (c) edge[loop left] node{$\emptyset, \{p\}$} (c);
			\draw (c) edge[->] node[below]{$\{p\}$} (d);
			\draw (d) edge[loop above] node{$\{p\}$} (d);
			\draw (d) edge[->, bend left = 15] node[above]{$\emptyset, \{p\}$} (e);
			\draw (e) edge[->, bend left = 15] node[below]{$\{p\}$} (d);
			\draw (e) edge[loop right] node{$\emptyset, \{p\}$} (e);
		\end{tikzpicture}
	\end{adjustbox}
		%\caption{After the Miyano-Hayashi construction}
	\end{figure}
\end{frame}

\begin{frame}{Satisfiability-Checking of LTL}
	\begin{itemize}
		\item Recall : The satisfiability problem for LTL asks, given an LTL formula $\phi$, if $\llbracket \phi \rrbracket$ is empty.
	\item Equivalent to check the emptiness of $\text{MH}(\mathcal{A}_{\phi})$.
	\item $\llbracket \phi \rrbracket = \text{L}_\text{b}(\mathcal{A}_{\phi}) = \emptyset$ iff $I^{\text{MH}}\cap\mathcal{F}_{\phi} = \emptyset$ where $\mathcal{F}_{\phi}$ is
	$$\mathcal{F}_{\phi} \equiv \nu y \cdot \mu x \cdot (\text{Pre}(x) \cup (\text{Pre}(y) \cap \alpha^{\text{MH}})) $$
	where $\text{Pre}(L) = \{q \in Q \mid \exists \sigma \in \Sigma \cdot \exists q' \in L : \sigma \cup \{q'\} \models \delta^{\text{MH}}(q)\}$.
	\item We call $\mathcal{F}_{\phi}$ a \textcolor{red}{backward} algorithm.
	\end{itemize}
\end{frame}

\begin{frame}{Satisfiability-Checking of LTL(cont'd)}
	\begin{itemize}
		\item The following fixpoint formulas compute the accepting reachable states $R_{\alpha}$ and then the set $\mathcal{F}'_{\phi}$ in a \textcolor{red}{forward} fashion.

		$$R_{\alpha} \equiv \alpha^{\text{MH}} \cap \mu x \cdot (\text{Post}(x) \cup I^{\text{MH}})$$
		$$\mathcal{F}'_{\phi} \equiv \nu y \cdot \mu x \cdot (\text{Post}(x) \cup (\text{Post}(y) \cap R_{\alpha})) $$
		
		where $\text{Post}(L) = \{q \in Q \mid \exists \sigma \in \Sigma \cdot \exists q' \in L : \sigma \cup \{q\} \models \delta^{\text{MH}}(q')\}$.
	\end{itemize}
	\begin{theorem}
		$\text{L}_{\text{b}}(\mathcal{A}_{\phi}) = \emptyset ~ iff ~ \mathcal{F}_{\phi} = \emptyset$.
	\end{theorem}
\end{frame}

\begin{frame}{Proof}
	\begin{itemize}
		\item Define $\Delta^{\text{MH}} : Q \times \Sigma^{+}$ the extension of the transition relation $\delta^{\text{MH}}$(recursively)
		\begin{itemize}
			\item $\Delta^{\text{MH}}(q, \sigma) = \delta^{\text{MH}}(q, \sigma)$
			\item $\Delta^{\text{MH}}(q, \mathnormal{w}\sigma) = \{q' \in Q \mid \exists q'' \in \Delta^{\text{MH}}(q, \mathnormal{w}) : \sigma \cup \{q'\} \models \delta^{\text{MH}}(q'')\}$ for each $q \in Q$, $\mathnormal{w} \in \Sigma^{+}$ and $\sigma \in \Sigma$.
		\end{itemize}
		\item Looping states $\mathcal{C}^{\text{MH}} = \{q \in Q \mid \exists \mathnormal{w} \in \Sigma^{+} : q \in \Delta^{\text{MH}}(q,\mathnormal{w})\}$
		\item From the definition of Büchi acceptance condition for NBW, we have $\text{L}_{\text{b}}(\mathcal{A}_{\phi}) = \emptyset$ iff $\mathcal{C}^{\text{MH}} \cap R_{\alpha} = \emptyset$.
	\end{itemize}
\end{frame}

\begin{frame}{Proof(cont'd)}
	\begin{itemize}
		\item Let $\text{HM}(\mathcal{A}_{\phi})$ be the reverse automaton.
		\item The following equivalences establish the theorem:
		\begin{itemize}
				\item $\nu y \cdot \mu x \cdot (\text{Pre}(x) \cup (\text{Pre}(y) \cap R_{\alpha})) = \emptyset$
				\item iff $\mathcal{C}^{\text{MH}} \cap R_{\alpha} = \emptyset$
				\item iff $\mathcal{C}^{\text{HM}} \cap R_{\alpha} = \emptyset$
				\item iff $\nu y \cdot \mu x \cdot (\text{Post}(x) \cup (\text{Post}(y) \cap R_{\alpha})) = \emptyset$
				\item iff $\mathcal{F}'_{\phi} = \emptyset$
		\end{itemize}
	\end{itemize}
\end{frame}

\begin{frame}{Closed Sets and Antichains}
	\begin{itemize}
		\item Let $\preceq \subseteq Q \times Q$ be a preorder.
		\item $q_{1} \prec q_{2}$ iff $q_{1} \preceq q_{2}$ and $q_{2} \npreceq q_{1}$
		\item A set $R \subseteq Q$ is \textcolor{red}{$\preceq$-closed} iff for all $q_{1}, q_{2} \in Q$, if $q_{1} \preceq q_{2}$ and $q_{2} \in R$ then $q_{1} \in R$.
		\item The \textcolor{red}{$\preceq$-closure} of $R$, is the set $\llbracket R \rrbracket = \{q \in Q \mid \exists q' \in R : q \preceq q'\}$.
		\item Let $\lceil R \rceil_{\preceq} = \{q \in R \mid \nexists q' \in R : q \prec q'\}$ be the set of \textcolor{red}{$\preceq$-maximal elements} of $R$.
		\item Let $\lfloor R \rfloor_{\succeq} = \{q \in R \mid \nexists q' \in R : q \succ q'\}$ be the set of \textcolor{red}{$\succeq$-minimal elements} of $R$.
		\item For all $\preceq$-closed sets $R \subseteq Q$, $R = \llbracket \lceil R \rceil_{\preceq} \rrbracket_{\preceq}$.
		\item For all $\succeq$-closed sets $R \subseteq Q$, $R = \llbracket \lfloor R \rfloor_{\succeq} \rrbracket_{\succeq}$.
		\item if $\preceq$ is a partial order, then $\lceil R \rceil_{\preceq}$ is an \textcolor{red}{antichain}.
	\end{itemize}
\end{frame}

\begin{frame}{Simulation}
	\begin{itemize}
		\item Let $\mathcal{A} = \langle \text{Loc}, I, \Sigma, \delta, \alpha \rangle$ be a NBW.
		\item A preorder $\preceq \subseteq \text{Loc} \times \text{Loc}$ is a \textcolor{red}{forward-simulation} for $\mathcal{A}$ if for all $q_{1}, q_{2}, q_{3} \in \text{Loc}$, for all $\sigma \in \Sigma$,
		\begin{itemize}
			\item if $q_{1} \preceq q_{2}$ and $q_{2} \xrightarrow{\sigma}_{\delta} q_{3}$ then there exists $q_{4} \in \text{Loc}$ such that $q_{1} \xrightarrow{\sigma}_{\delta} q_{4}$ and $q_{4} \preceq q_{3}$, and
			\item if $q_{1} \preceq q_{2}$ and $q_{2} \in \alpha$ then $q_{1} \in \alpha$.
		\end{itemize}
	\end{itemize}

\tikzset{
%->, % makes the edges directed
>=stealth, % makes the arrow heads bold
every edge/.style={draw, thick, black},
node distance=0.8cm and 2.3cm, % specifies the minimum distance between two nodes. Change if necessary.
every state/.style={thick, fill=gray!10}, % sets the properties for each ’state’ node
initial text=$ $, % sets the text that appears on the start arrow
}
	\begin{figure}[ht] % ’ht’ tells LaTeX to place the figure ’here’ or at the top of the page
	\centering % centers the figure
	\begin{adjustbox}{max width=0.9\textwidth, max totalheight=\textheight}
		\begin{tikzpicture}
			\node[] (then) {then};
			\node[state, above left = of then] (l2) {$q_{3}$};
			\node[state, left = of l2] (l1) {$q_{2}$};
			\node[below = of l1] (mid) {
			\begin{tikzpicture}
      		\node [rotate=90] {$\preceq$};    
    		\end{tikzpicture}
};
			\node[state, below = of mid] (l3) {$q_{1}$};
			\node[below left = of l1] (if) {If};
			
			\node[state, below right = of then] (l3') {$q_{1}$};
			\node[state, right = of l3'] (l4') {$q_{4}$};
			\node[above = of l4'] (mid') {
			\begin{tikzpicture}
      		\node [rotate=90] {$\preceq$};    
    		\end{tikzpicture}
};
			\node[state, above = of mid'] (l2') {$q_{3}$};

			\draw (l1) edge[->] node[above] {$\sigma$} (l2);
			\draw (l3') edge[->] node[above] {$\sigma$} (l4');

			\draw (l1) edge[dashed] (mid);
			\draw (mid) edge[dashed] (l3);
			\draw (l4') edge[dashed] (mid');
			\draw (mid') edge[dashed] (l2');
		\end{tikzpicture}
	\end{adjustbox}
		\caption{Simulation}
	\end{figure}
\end{frame}
\end{document}