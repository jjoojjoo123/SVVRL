\documentclass[12pt]{beamer}
\usepackage{listings}
\usepackage{color}
\usepackage{xcolor}
%\usepackage{latexsym}
\usepackage{amsmath}
\usepackage[labelfont=bf]{caption}
\usepackage{graphicx}  %package graphic
\usepackage{siunitx}
\usepackage{tikz}
%\usepackage{algorithmicx}
%\usepackage[noend]{algpseudocode}
\usetikzlibrary{automata, positioning, arrows}
\usetheme{Boadilla}   
\usepackage{xeCJK}
%\usepackage{array}
%\usepackage{tabularx}
%\usepackage{mathtools}
\usepackage{listings}
%\usepackage{textcomp}
\usepackage[T1]{fontenc}
\usepackage{lmodern}
\usepackage{adjustbox}
\usepackage{dirtree}

\setCJKmainfont{微軟正黑體} 
\sisetup{
    group-separator={,},
    table-number-alignment=right
}


\setbeamerfont{title}{size=\Large,series=\bfseries}  % title size

\setbeamerfont{frametitle}{size=\large,series=\bfseries}  % frametitle size, also can size*=<pt>

\definecolor{themeColor}{rgb}{0.06, 0.3, 0.57}
\usecolortheme[named=themeColor]{structure}

\defbeamertemplate*{footline}{Dan P theme}
{
  \leavevmode%
  \hbox{%
  %\begin{beamercolorbox}[wd=.2\paperwidth,ht=2.25ex,dp=1ex,center]{author in head/foot}%
    %\usebeamerfont{author in head/foot}\insertshortauthor\expandafter\beamer@ifempty\expandafter{\beamer@shortinstitute}{}{~~(\insertshortinstitute)}
  %\end{beamercolorbox}%
  \begin{beamercolorbox}[wd=.7\paperwidth,ht=2.25ex,dp=1ex,center]{title in head/foot}%
    \usebeamerfont{title in head/foot}\insertshorttitle
  \end{beamercolorbox}%
  \begin{beamercolorbox}[wd=.3\paperwidth,ht=2.25ex,dp=1ex,right]{date in head/foot}%
    \usebeamerfont{date in head/foot}\insertshortdate{}\hspace*{2em}
\insertframenumber{} / \inserttotalframenumber\hspace*{2ex} 
  \end{beamercolorbox}}%
  \vskip0pt%
}
\makeatother

% Item include picture
\setbeamertemplate{itemize item}   % First Level item
{\includegraphics[height=0.33cm]{../Figures/golden-earth-on-white}}

\setbeamertemplate{itemize subitem} % Second level item
{\includegraphics[height=0.31cm]{../Figures/golden-sun-on-white}}

\setbeamertemplate{itemize subsubitem} % Third Level item
{\includegraphics[height=0.27cm]{../Figures/golden-paw-on-white}}

\definecolor{darkgold}{rgb}{0.765 0.64 0.0} % for highlighted text in black-and-white slides
\newcommand{\highlight}[1]{\structure{#1}}
\newcommand{\highlightb}[1]{\textcolor{blue}{#1}}
\newcommand{\highlightg}[1]{\textcolor{darkgold}{#1}}
\newcommand{\highlightr}[1]{\alert{#1}}
\newcommand{\code}[1]{\texttt{#1}}

%\newtheorem{problem}{Problem}
\mode<presentation>{\newtheorem{algorithm}{Algorithm}}
\mode<article>{\newenvironment{algorithm}{}{}}
%\newtheorem{solution}{Solution}

\newcommand{\hide}[1]{}
%\renewcommand{\highlightb}{\highlightg}

\newlength{\subtextwidth}
\setlength{\subtextwidth}{11cm}

\newenvironment{cbox}{
    \begin{center}
    \begin{tabular}{|l|}
    \hline
    \begin{minipage}[t]{\subtextwidth}}
    {\vspace{.25ex}
    \end{minipage}
    \hline
    \end{tabular}
    \end{center} }

\lstdefinestyle{mystyle}{
    basicstyle=\setmonofont{Consolas},
    columns=fullflexible,
    keepspaces=true,
    upquote=true,
    showstringspaces=false,
    commentstyle=\color{gray},
    keywordstyle=\color{teal},
    identifierstyle=\color{black},
    stringstyle=\color{brown},
    language=C++
}

\lstset{style=mystyle,frame=single,}
\makeatletter
\lst@CCPutMacro
    \lst@ProcessOther {"2A}{%
    %   \lst@ttfamily 
         {\raisebox{1.5pt}{*}}% used with ttfamily
        %   \textasteriskcentered
          }% used with other fonts
    \@empty\z@\@empty
\makeatother

\newcommand{\mathdef}[1]{\relax\ifmmode #1\else $#1$\fi}
\newcommand{\true}{\mathdef{\mathit{true}}}
\newcommand{\false}{\mathdef{\mathit{false}}}
\renewcommand{\implies}{\mathdef{\rightarrow}}
\newcommand{\ifonlyif}{\mathdef{\leftrightarrow}}
\newcommand{\entails}{\mathdef{\vdash}}
\newcommand{\PROPS}{\mathdef{\mathit{PROPS}}}
\newcommand{\BOOL}{\mathdef{\mathit{BOOL}}}
\newcommand{\Note}[1]{\textsl{\small(#1)}}

\mode<presentation>{\title{DIVINE Code Tracing}}
\mode<article>{\title{Algorithms 2019: Analysis of Algorithms}}
%\subtitle{(Based on [Manber 1989])}
\author{林宏陽}
%\institute[IM.NTU]{Department of Information Management\\ National Taiwan University}
%\date[Algorithms 2019]{\null}
\mode<presentation>{\date[SVVRL]{\null}}
\mode<article>{\date{\today}}

\begin{document}
\begin{frame}
\maketitle
\end{frame}

\begin{frame}{DIVINE's source file structure}
    \dirtree{%
        .1 divine-4.4.2.
        .2 bricks	\Note{the bricks C++ library used for threading, command line parsing and a couple of other utilities}.
        .2 clang.
        .2 compiler-rt.
        .2 cryptoms.
        .2 dios.
        .2 divine	\Note{contains sources and headers of the verification \& state space generation framework and actual model checking algorithms}.
        .2 doc		\Note{out-of-date}.
        .2 \dots.
    }
\end{frame}

\begin{frame}{DIVINE's source file structure (cont'd)}
    \dirtree{%
        .1 .
        .2 external		\Note{the external code used in DIVINE such as the libc++ library, the murphi compiler for divine, or the parallel allocators}.
        .2 lart.
        .2 lld.
        .2 llvm.
        .2 minisat.
        .2 releng.
        .2 stp.
        .2 test.
        .2 tools	\Note{is for binaries}.
        .3 \dots.
        .3 divine.cpp	\Note{the main user-level entry point of the tool}.
        .3 \dots.
    }
\end{frame}

\begin{frame}{DIVINE's source file structure (cont'd)}
    \dirtree{%
        .1 divine-4.4.2.
        .2 divine.
        .3 cc	\Note{C/C++ compiler}.
        .3 dbg	\Note{Debug}.
        .3 ltl.
        .3 mc	\Note{Model Checking}.
        .3 mem.
        .3 ra.
        .3 rt	\Note{Runtime}.
        .3 sim	\Note{Simulation}.
        .3 smt	\Note{SMT solver}.
        .3 ss	\Note{State Space}.
        .3 ui	\Note{User Interface}.
        .3 vm.
    }
\end{frame}


\begin{frame}[fragile]{tools/divine.cpp}
\begin{lstlisting}[basicstyle=\footnotesize\ttfamily]
#include <divine/ui/cli.hpp>

int main( int argc, const char **argv )
{
    if ( argc > 1 && ( !strcmp( argv[ 1 ], "--help" ) 
    || !strcmp( argv[ 1 ], "--version" ) ) )
            argv[ 1 ] += 2;

    auto ui = divine::ui::make_cli( argc, argv )->resolve();
    divine::ui::setup_death( ui );
    return ui->main();
}
\end{lstlisting}
Note : \code{cli} for Command Line Interface.
\end{frame}

\begin{frame}[fragile, label=parser.hpp]{ui/parser.hpp}
\begin{lstlisting}[basicstyle=\footnotesize\ttfamily]
struct CLI : Interface
{
    /* ... */
    virtual int main() override try
    {
        auto cmd = parse();
        cmd.match( [&]( brq::cmd_help &help ) { help.run();},
                   [&]( command &c )
                   {
                       c.setup();
                       c.run();
                       c.cleanup();
                   } );
        return 0;
    }
    /* ... */
};
\end{lstlisting}
Note : \code{cmd} for command, \code{brq} for brick.
\end{frame}

\begin{frame}[fragile]{ui/cli.hpp}
\begin{lstlisting}[basicstyle=\footnotesize\ttfamily]
namespace divine::ui
{
    struct command : brq::cmd_base {/* ... */};
    struct with_bc : command {/* ... */};
    struct version : command {/* ... */};
    struct ltlc : command {/* ... */};
    struct with_report : with_bc
    {
        /* ... */
        void setup_report_file();
        void cleanup() override;
        void options( brq::cmd_options &c ) override {/* ... */}
    };
    /* ... */
}
\end{lstlisting}
Note : \code{bc} for bit code.
\end{frame}

\begin{frame}[fragile]{ui/cli.hpp}
\begin{lstlisting}[basicstyle=\footnotesize\ttfamily]
namespace divine::ui {
    struct verify : with_report
    {
        /* ... */
        void setup() override;
        void run() override;
        void safety();
        void liveness();
        void print_ce( mc::Job &job );
        std::string_view help() override {/* ... */}
        void options( brq::cmd_options &c ) override {
            with_report::options( c );
            c.section( "Verification Options" );
            c.opt( "--threads", _threads )
            << "number of worker threads to use";
            /* ... */
        }
    };
}
\end{lstlisting}
\end{frame}

\begin{frame}[fragile]{ui/cli.hpp}
\begin{lstlisting}[basicstyle=\footnotesize\ttfamily]
namespace divine::ui {
    struct check : verify
    {
        void setup() override;
        std::string_view help() override {/* ... */}
    };
    struct exec : with_report {/* ... */};
    struct sim : with_bc {/* ... */};
    struct draw : with_bc {/* ... */};
    struct refine : with_bc {/* ... */};
    struct cc : command {/* ... */};
    struct info : exec {/* ... */};
}
\end{lstlisting}
\end{frame}

\againframe{parser.hpp}

\begin{frame}[fragile]{ui/verify.cpp}
\begin{lstlisting}[basicstyle=\footnotesize\ttfamily]
void verify::setup()
{
    if ( _log == nullsink() )
    {
        std::vector< SinkPtr > log;
        if ( _interactive )
            log.push_back( make_interactive() );
        if ( _report != arg::report::none )
            log.push_back( make_yaml( std::cout
            , _report == arg::report::yaml_long ) );
        /* ... */
    }
    if ( _bc_opts.dios_config.empty() && _liveness )
        _bc_opts.dios_config = "fair";
    with_bc::setup();
    if ( _bc_opts.symbolic )
        bitcode()->solver( _solver );
}
\end{lstlisting}
\end{frame}

\begin{frame}[fragile]{ui/cli.cpp}
\begin{lstlisting}[basicstyle=\footnotesize\ttfamily]
void with_bc::setup()
{
    using bstr = std::vector< uint8_t >;
    process_options();
    int i = 0;
    std::set< std::string > vfsCaptured;
    size_t limit = _vfs_limit.size;
    /* ... */
    _bc = mc::BitCode::with_options( _bc_opts, _cc_driver );
}
\end{lstlisting}
Note : \code{vfs} for Virtual File System
\end{frame}

\begin{frame}[fragile]{mc/bitcode.cpp}
\begin{lstlisting}[basicstyle=\footnotesize\ttfamily]
std::shared_ptr< BitCode > BitCode::with_options
( const BCOptions &opts, rt::DiosCC &cc_driver )
{
    auto magic_buf = 
    cc_driver.compiler.getFileBuffer( opts.input_file.name, 18 );
    auto magic_data = magic_buf ?
        std::string( magic_buf->getBuffer() )
        : brq::read_file( opts.input_file.name, 18 );
    auto magic = llvm::identify_magic( magic_data );
    
    auto bc = [&]{
        /* ... */
        cc_driver.build( cc::parseOpts( opts.ccopts ) );
        // Compile all the files and link them together
        // , including necessary libraries
    }();
    bc->set_options( opts );
    return bc;
}
\end{lstlisting}
\end{frame}

\againframe{parser.hpp}

\begin{frame}[fragile, label=safety_liveness]{ui/verify.cpp}
\begin{lstlisting}[basicstyle=\normalsize\ttfamily]
void verify::run()
{
    if ( _liveness )
        liveness();
    else
        safety();
}
\end{lstlisting}
\end{frame}

\begin{frame}[fragile]{ui/verify.cpp}
\begin{lstlisting}[basicstyle=\footnotesize\ttfamily]
void verify::safety()
{
    auto safety = mc::make_job< mc::Safety >( bitcode()
    , ss::passive_listen() );
    /* ... */
    safety->start( _threads, [&]( bool last ){
        /* ... */
        if ( last || ( ++ps_ctr == 2 * _poolstat_period ) )
            ps_ctr = 0, _log->memory( safety->poolstats()
            , safety->hashstats(), last );
        if ( !last )
            sysinfo.updateAndCheckTimeLimit( _max_time );
        } );
    safety->wait();
    report_options();
    /* ... */
    print_ce( *safety );
}
\end{lstlisting}
\end{frame}

\begin{frame}[fragile]{mc/safety.hpp}
\begin{lstlisting}[basicstyle=\footnotesize\ttfamily]
template< typename Next, typename Builder > struct Safety : Job{
    void start( int threads ) override {
        using Search = decltype( make_search() );
        _search.reset( new Search( std::move( make_search() ) ) );
        Search *search = dynamic_cast< Search * >( _search.get() );
        stats = [=](){
            int64_t st = _ex._d.total_states->load();
            int64_t mip = _ex._d.total_instructions->load();
            search->ws_each( [&]( auto &bld, auto & ){
                st += bld._d.local_states;
                mip += bld._d.local_instructions;
            } );
            return std::make_pair( st, mip );};
        queuesize = [=]() { return search->qsize(); };
        search->start( threads );
    }
}
\end{lstlisting}
Note : \code{bld} for builder
\end{frame}

\begin{frame}[fragile]{ss/search.hpp (for safety properties)}
\begin{lstlisting}[basicstyle=\footnotesize\ttfamily]
template< typename B, typename L > struct Search : Job
{
    void start( int thread_count ) override
    {
        _thread_count = thread_count;
        Worker blueprint;

        switch ( _order )
        {
            case Order::PseudoBFS: blueprint = pseudoBFS(); break;
            case Order::DFS: blueprint = DFS(); break;
        }

        for ( int i = 0; i < _thread_count; ++i )
            _threads.emplace_back( std::async( blueprint ) );
    }
}
\end{lstlisting}
\end{frame}

\begin{frame}[fragile]{ss/search.hpp (for safety properties)}
\begin{lstlisting}[basicstyle=\footnotesize\ttfamily]
struct DFSItem
{
    enum Type { Pre, Post } type;
    State state;
    Label label;
    DFSItem( Type t, State s, Label l = Label() )
        : type( t ), state( s ), label( l ) {}
};
\end{lstlisting}
\begin{lstlisting}[basicstyle=\footnotesize\ttfamily]
Worker DFS()
{
    auto builder = _builder;
    auto listener = _listener;
    return [=]() mutable
    {
        auto _reg = _register( builder, listener );
        std::stack< DFSItem > stack;
        builder.initials( 
            [&]( auto st ){ stack.emplace( DFSItem::Pre, st ); });
\end{lstlisting}
\end{frame}

\begin{frame}[fragile]{ss/search.hpp (for safety properties)}
\begin{lstlisting}[basicstyle=\footnotesize\ttfamily]
        while ( !stack.empty() && !_terminate->load() ){
            auto top = stack.top(); stack.pop();
            if ( top.type == DFSItem::Post )
                listener.closed( top.state );
            else _state( listener, top.state, true,
                [&]( bool )
                {
                    stack.emplace( DFSItem::Post, top.state );
                    _succs( listener, builder, top.state,
                        [&]( auto s, auto l, bool )
                        { stack.emplace( DFSItem::Pre, s, l ); }
                        );
                } );
        }
    };
}
\end{lstlisting}
\end{frame}

\begin{frame}{LTL verification is not avalible in current version}
    DIVINE 3.90
    \begin{itemize}
        \item Substantial rewrite with focus on software verification.
        \item $\cdots$
    \end{itemize}
    LIMITATIONS: This is a beta version. Additionally, verification of \textbf{LTL (liveness) properties} and distributed verification are \textbf{not currently available} and are planned for a later version in the 4.x series. Only C and C++ models are supported at this time: DVE, DESS, CESMI, UPPAAL and other input formalisms are not supported, but may become available in later versions.
\end{frame}

\begin{frame}{The meaning of "symbolic" in DIVINE}
SymDIVINE is motivated as an extension of our purely explicit model
checker DIVINE that is capable of handling full parallel C/C++ programs without inputs. To properly handle programs with inputs, SymDIVINE relies on Control-Explicit Data-Symbolic approach.

\vspace{2cm}

Note : SymDIVINE functionality was integrated into DIVINE.
\end{frame}

\begin{frame}{The meaning of "symbolic" in DIVINE (cont'd)}
The code represents a simple LLVM program, where \code{a} is initialized with a non-deterministic 32-bit integer, then it is checked whether it is greater or equal to 65535. The result of the check is stored to \code{b} and used for branching.\\
\vspace{1.5cm}
\code{\%a = call i32 \@\_\_VERIFIER\_nondet\_int()}\\
\code{\%b = icmp sge i32 \%a, 65535}\\
\code{br i1 \%b, label \%5, label \%6}
\end{frame}

\captionsetup[figure]{font=scriptsize ,labelfont=scriptsize}
\begin{frame}{The meaning of "symbolic" in DIVINE (cont'd)}
    \begin{figure}
        \centering
        \includegraphics[scale=0.3]{symDIVINE.png}
        \caption{The figure compares state exploration in the explicit approach of DIVINE and in the control-explicit data-symbolic approach of SymDIVINE on LLVM program example. From init state DIVINE explores states for every possible value of a ($2^{32}$ values), hence exponentially expands state space. In contrast SymDIVINE approach of symbolic representation generates only two different states. One where the condition on branching ($a \geq 65535$) is satisfied and the other one where the condition is violated.}
    \end{figure}    
\end{frame}

\begin{frame}{The meaning of "symbolic" in DIVINE (cont'd)}
    \begin{itemize}
        \item While a purely explicit-state model checker has to produce a new state for each and every possible input value, in SymDIVINE a set of states that differ only in data values is represented with a single data structure, the so called \textbf{multi-state}.
        \item In the current version of SymDIVINE, only quantified bit-vector SMT formulae are supported to represent symbolic data.
        \item SymDIVINE employs an SMT solver and quantified formulae to check the satisfiability of a path condition and to decide the equality of two multi-states.
    \end{itemize}
\end{frame}

\againframe{safety_liveness}

\begin{frame}[fragile]{mc/verify.cpp}
\begin{lstlisting}[basicstyle=\footnotesize\ttfamily]
void verify::liveness()
{
    auto liveness = mc::make_job< mc::Liveness >( bitcode(),
    ss::passive_listen() );

    _log->start();
    liveness->start( 1, [&]( bool last )
            {
                _log->progress( liveness->stats(),
                                liveness->queuesize(), last );
            } );
    liveness->wait();
    report_options();
    _log->info( "property type: liveness\n", true );
    print_ce( *liveness );
}
\end{lstlisting}
\end{frame}

\begin{frame}[fragile]{mc/liveness.hpp}
\begin{lstlisting}[basicstyle=\footnotesize\ttfamily]
template< typename Next, typename Builder_ = ExplicitBuilder >
struct Liveness : Job
{
    void start( int threads ) override
    {
        auto *search = new NestedDFS( _ex );
        _search.reset( search );
        stats = [=] { return std::pair( _ex._d.total_states->load()
            , _ex._d.total_instructions->load() ); };
        queuesize = [=] { return search->outer_stack.size() 
            + search->inner_stack.size(); };

        /* ... ( definition of function _get_trace ) */ 
        _error_found = [=]() { return 
            search->counterexample.goal.has_value(); };

        search->start( threads );
    }
}
\end{lstlisting}
\end{frame}

\begin{frame}[fragile]{mc/liveness.hpp}
\begin{lstlisting}[basicstyle=\footnotesize\ttfamily]
template< typename Builder >
struct NestedDFS : ss::Job
{
    std::future< void > _thread;
    void start( int thread_count ) override {
        if ( thread_count != 1 )
            throw new std::runtime_error( 
                "Nested DFS only supports one thread." );
        _thread = std::async( [&]{ run(); } );
    }
    void run() {
        _builder.initials( 
            [found = false, this] ( State state ) mutable
            {
                if ( !found )
                    found = outer( state );
            } );
    }
}
\end{lstlisting}
\end{frame}


\begin{frame}{Reference}
\begin{itemize}
    \item DIVINE website\\
        \code{https://divine.fi.muni.cz/}
    \item Jan Mrázek, Petr Bauch, Henrich Lauko, Jiří Barnat: SymDIVINE: Tool for Control-Explicit Data-Symbolic State Space Exploration. In Model Checking Software: 23rd International Symposium, SPIN, Springer International Publishing, 2016.
\end{itemize}
\end{frame}


\end{document}