\documentclass[12pt]{beamer}
\usepackage{listings}
\usepackage{color}
\usepackage{xcolor}
\usepackage{latexsym}
\usepackage{amsmath}
\usepackage[labelfont=bf]{caption}
\usepackage{graphicx}  %package graphic
\usepackage{siunitx}
\usepackage{tikz}
\usepackage{algorithmicx}
\usepackage[noend]{algpseudocode}
\usetikzlibrary{automata, positioning, arrows}
\usetheme{Boadilla}   
\usepackage{xeCJK}
\usepackage{array}
\usepackage{tabularx}
\usepackage{mathtools}


\setCJKmainfont{微軟正黑體} 
\sisetup{
	group-separator={,},
	table-number-alignment=right
}


\setbeamerfont{title}{size=\Large,series=\bfseries}  % title size

\setbeamerfont{frametitle}{size=\large,series=\bfseries}  % frametitle size, also can size*=<pt>

% Item include picture
\setbeamertemplate{itemize item}   % First Level item
{\includegraphics[height=0.33cm]{Figures/golden-earth-on-white}}

\setbeamertemplate{itemize subitem} % Second level item
{\includegraphics[height=0.31cm]{Figures/golden-sun-on-white}}

\setbeamertemplate{itemize subsubitem} % Third Level item
{\includegraphics[height=0.27cm]{Figures/golden-paw-on-white}}

\definecolor{darkgold}{rgb}{0.765 0.64 0.0} % for highlighted text in black-and-white slides
\newcommand{\highlight}[1]{\structure{#1}}
\newcommand{\highlightb}[1]{\textcolor{blue}{#1}}
\newcommand{\highlightg}[1]{\textcolor{darkgold}{#1}}
\newcommand{\highlightr}[1]{\alert{#1}}
\newcommand{\code}[1]{\texttt{#1}}

%\newtheorem{problem}{Problem}
\mode<presentation>{\newtheorem{algorithm}{Algorithm}}
\mode<article>{\newenvironment{algorithm}{}{}}
%\newtheorem{solution}{Solution}

\newcommand{\hide}[1]{}
%\renewcommand{\highlightb}{\highlightg}

\newlength{\subtextwidth}
\setlength{\subtextwidth}{11cm}

\newenvironment{cbox}{
    \begin{center}
    \begin{tabular}{|l|}
    \hline
    \begin{minipage}[t]{\subtextwidth}}
    {\vspace{.25ex}
    \end{minipage}
    \hline
    \end{tabular}
    \end{center} }

\newcommand{\mathdef}[1]{\relax\ifmmode #1\else $#1$\fi}
\newcommand{\true}{\mathdef{\mathit{true}}}
\newcommand{\false}{\mathdef{\mathit{false}}}
\renewcommand{\implies}{\mathdef{\rightarrow}}
\newcommand{\ifonlyif}{\mathdef{\leftrightarrow}}
\newcommand{\entails}{\mathdef{\vdash}}
\newcommand{\PROPS}{\mathdef{\mathit{PROPS}}}
\newcommand{\BOOL}{\mathdef{\mathit{BOOL}}}

\mode<presentation>{\title{SPIN's CheckEmpty Procedure\\ Code Tracing}}
\mode<article>{\title{Algorithms 2019: Analysis of Algorithms}}
%\subtitle{(Based on [Manber 1989])}
\author{林宏陽}
%\institute[IM.NTU]{Department of Information Management\\ National Taiwan University}
%\date[Algorithms 2019]{\null}
\mode<presentation>{\date[SVVRL]{\null}}
\mode<article>{\date{\today}}

\begin{document}
\begin{frame}
\maketitle
\end{frame}

\begin{frame}{Model Checking Using Automata}
	\begin{itemize}
	\item The given system is modeled as a Büchi automaton $A$.
	\item Suppose the desired property is originally given by a linear temporal formula $f$.
	\item Let $B_{f}$ (resp. $B_{\neg f}$) denote a Büchi automaton equivalent to $f$ (resp. $\neg f$ ).
	\item The model checking problem $A \models f$ is equivalent to asking whether
$$L(A) \subseteq L(B_{f}) \quad or \quad L(A) \cap L(B_{\neg f}) = \emptyset.$$
	\item The well-used model checker SPIN, for example, adopts this automata-theoretic approach.
	\item So, we are left with two basic problems:
			\begin{itemize}
			\item Compute the intersection of two Büchi automata.
			\item Test the emptiness of the resulting automaton.
			\end{itemize}
	\end{itemize}
\end{frame}

\begin{frame}{Checking Emptiness}
	\begin{itemize}
	\item Checking nonemptiness of $L(B)$ is equivalent to finding a strongly connected component that is reachable from an initial state and contains an accepting state.
	\item That is, the language $L(B)$ is nonempty iff \textcolor{blue}{there is a reachable accepting state with a cycle back to itself}.
	\item If we find such a cycle when checking the emptiness of $L(A) \cap L(B_{\neg f})$ , bad news, $A \not\models f$.
	\item Otherwise, good news, $A \models f$.
	\end{itemize}
\end{frame}

\begin{frame}{Double DFS Algorithm}
\only<1>{\tikzset{
->, % makes the edges directed
>=stealth, % makes the arrow heads bold
every edge/.style={draw, thick, black},
node distance=2.7cm, % specifies the minimum distance between two nodes. Change if necessary.
every state/.style={thick, fill=gray!10}, % sets the properties for each ’state’ node
initial text=$ $, % sets the text that appears on the start arrow
}}

\only<2-3>{\tikzset{
->, % makes the edges directed
>=stealth, % makes the arrow heads bold
every edge/.style={draw, thick, cyan},
node distance=2.7cm, % specifies the minimum distance between two nodes. Change if necessary.
every state/.style={thick, fill=gray!10}, % sets the properties for each ’state’ node
initial text=$ $, % sets the text that appears on the start arrow
}}

	\begin{figure}[ht] % ’ht’ tells LaTeX to place the figure ’here’ or at the top of the page
	\centering % centers the figure
		\begin{tikzpicture}
			\node[state, initial] (1) {};
			\node[state, above right of=1] (2) {};
			\node[state, right of=2] (3) {};
			\node[state, right of=1] (4) {};
			\node[state, right of=4] (5) {};
			\node[state, below right of=1] (6) {};
			\node[state, right of=6] (7) {};
			\node[state, accepting, right of=7] (8) {};
			\node[state, below right of=7] (9) {};
			\node[state, right of=9] (10) {};
			\draw (1) edge node{} (2)
						(2) edge node{} (3)
						(1) edge node{} (4)
						(4) edge node{} (5)
						(1) edge node{} (6)
						(6) edge node{} (7)
						(7) edge node{} (8)
						(8) edge node{} (10)
						(10) edge node{} (9)
						(9) edge[black] node{} (7);
						\only<3>{
						\draw (8) edge[red] node{} (10)
						(10) edge[red] node{} (9)
						(9) edge[red] node{} (7);}
		\end{tikzpicture}
		\only<2>{\caption{DFS Tree}}
		\only<3>{\caption{When returning to an accepting state, it starts the second DFS (post-order)}}
	\end{figure}
\end{frame}

\begin{frame}{SPIN's structure}
\fontsize{6pt}{7.2}\selectfont
	\begin{figure}[ht]
	\centering
	\begin{tikzpicture}
		\tikzset{every node/.style={draw, rectangle, align=center, minimum height=0.8cm, minimum width=1.6cm, inner sep=2pt},
		arrow/.style = {thick,>=stealth}}

			\node(n1) {XSPIN\\Front-End\\(Tcl/Tk Code)};
			\node[below = of n1](n2) {PROMELA\\Parser};
			\node[right = of n2](n3) {LTL Parser\\and Translator};
			\node[below = of n2](n4) {2.\\Interactive\\Simulation};
			\node[left = of n4](n5) {1.\\Syntax Error\\Reports};
			\node[right = of n4](n6) {3.\\Verifier\\Generator};
			\node[below = of n6](n7) {Optimized\\Model Checker\\(ANSI C Code)};
			\node[below = of n7](n8) {Excutable\\On-The-Fly\\Verifier};
			\node[draw = none, left = 0cm of n8](n9) {Counter-\\Examples};
		
			\draw[arrow,<->] (n1) -- (n2);
			\draw[arrow,<->] (n2) -- (n3);
			\draw[arrow,->] (n2) -- (n4);
			\draw[arrow,->] (n2) -- (n5);
			\draw[arrow,->] (n2) -- (n6);
			\draw[arrow,->] (n6) -- (n7);
			\draw[arrow,->] (n7) -- (n8);
			\draw[arrow,->] (n8) -| (n4);
		\end{tikzpicture}
		%\caption{The structure of SPIN simulation and verification. Source: redrawn from [The SPIN Model Checker Figure 11.1 Gerard J. Holzmann]}
		\end{figure}
\end{frame}

\begin{frame}{If you want to get the code...}
	\begin{itemize}
		\item Download SPIN from the github and set it up.
		\item Suppose there is a system to verify described in PROMELA named \code{example.pml}.
		\item \code{\$ spin -a example.pml   \hspace{1cm}\#generate c-verifier}
		\item \code{\$ gcc -E -DNOREDUCE -DBITSTATE pan.c > ppan.c}
	\end{itemize}
\end{frame}


\end{document}