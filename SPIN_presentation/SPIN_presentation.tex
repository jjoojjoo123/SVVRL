\documentclass[12pt]{beamer}
\usepackage{listings}
\usepackage{color}
\usepackage{xcolor}
\usepackage{latexsym}
\usepackage{amsmath}
\usepackage[labelfont=bf]{caption}
\usepackage{graphicx}  %package graphic
\usepackage{siunitx}
\usepackage{tikz}
%\usepackage{algorithmicx}
%\usepackage[noend]{algpseudocode}
\usetikzlibrary{automata, positioning, arrows}
\usetheme{Boadilla}   
\usepackage{xeCJK}
\usepackage{array}
%\usepackage{tabularx}
\usepackage{mathtools}
\usepackage{listings}
\usepackage{textcomp}
\usepackage[T1]{fontenc}
\usepackage{lmodern}

\setCJKmainfont{微軟正黑體} 
\sisetup{
	group-separator={,},
	table-number-alignment=right
}


\setbeamerfont{title}{size=\Large,series=\bfseries}  % title size

\setbeamerfont{frametitle}{size=\large,series=\bfseries}  % frametitle size, also can size*=<pt>

% Item include picture
\setbeamertemplate{itemize item}   % First Level item
{\includegraphics[height=0.33cm]{../Figures/golden-earth-on-white}}

\setbeamertemplate{itemize subitem} % Second level item
{\includegraphics[height=0.31cm]{../Figures/golden-sun-on-white}}

\setbeamertemplate{itemize subsubitem} % Third Level item
{\includegraphics[height=0.27cm]{../Figures/golden-paw-on-white}}

\definecolor{darkgold}{rgb}{0.765 0.64 0.0} % for highlighted text in black-and-white slides
\newcommand{\highlight}[1]{\structure{#1}}
\newcommand{\highlightb}[1]{\textcolor{blue}{#1}}
\newcommand{\highlightg}[1]{\textcolor{darkgold}{#1}}
\newcommand{\highlightr}[1]{\alert{#1}}
\newcommand{\code}[1]{\texttt{#1}}

%\newtheorem{problem}{Problem}
\mode<presentation>{\newtheorem{algorithm}{Algorithm}}
\mode<article>{\newenvironment{algorithm}{}{}}
%\newtheorem{solution}{Solution}

\newcommand{\hide}[1]{}
%\renewcommand{\highlightb}{\highlightg}

\newlength{\subtextwidth}
\setlength{\subtextwidth}{11cm}

\newenvironment{cbox}{
    \begin{center}
    \begin{tabular}{|l|}
    \hline
    \begin{minipage}[t]{\subtextwidth}}
    {\vspace{.25ex}
    \end{minipage}
    \hline
    \end{tabular}
    \end{center} }

\lstdefinestyle{mystyle}{
    basicstyle=\ttfamily,
    columns=fullflexible,
    keepspaces=true,
    upquote=true,
    showstringspaces=false,
    commentstyle=\color{olive},
    keywordstyle=\color{blue},
    identifierstyle=\color{violet},
    stringstyle=\color{purple},
    language=c,
    directivestyle=\color{teal},
}

\lstset{style=mystyle,frame=single,}
\makeatletter
\lst@CCPutMacro
    \lst@ProcessOther {"2A}{%
      \lst@ttfamily 
         {\raisebox{2pt}{*}}% used with ttfamily
         \textasteriskcentered}% used with other fonts
    \@empty\z@\@empty
\makeatother

\newcommand{\mathdef}[1]{\relax\ifmmode #1\else $#1$\fi}
\newcommand{\true}{\mathdef{\mathit{true}}}
\newcommand{\false}{\mathdef{\mathit{false}}}
\renewcommand{\implies}{\mathdef{\rightarrow}}
\newcommand{\ifonlyif}{\mathdef{\leftrightarrow}}
\newcommand{\entails}{\mathdef{\vdash}}
\newcommand{\PROPS}{\mathdef{\mathit{PROPS}}}
\newcommand{\BOOL}{\mathdef{\mathit{BOOL}}}

\mode<presentation>{\title{SPIN's CheckEmpty Procedure\\ Code Tracing}}
\mode<article>{\title{Algorithms 2019: Analysis of Algorithms}}
%\subtitle{(Based on [Manber 1989])}
\author{林宏陽}
%\institute[IM.NTU]{Department of Information Management\\ National Taiwan University}
%\date[Algorithms 2019]{\null}
\mode<presentation>{\date[SVVRL]{\null}}
\mode<article>{\date{\today}}

\begin{document}
\begin{frame}
\maketitle
\end{frame}

\begin{frame}{Model Checking Using Automata}
	\begin{itemize}
	\item The given system is modeled as a Büchi automaton $A$.
	\item Suppose the desired property is originally given by a linear temporal formula $f$.
	\item Let $B_{f}$ (resp. $B_{\neg f}$) denote a Büchi automaton equivalent to $f$ (resp. $\neg f$ ).
	\item The model checking problem $A \models f$ is equivalent to asking whether
$$L(A) \subseteq L(B_{f}) \quad or \quad L(A) \cap L(B_{\neg f}) = \emptyset.$$
	\item The well-used model checker SPIN, for example, adopts this automata-theoretic approach.
	\item So, we are left with two basic problems:
			\begin{itemize}
			\item Compute the intersection of two Büchi automata.
			\item Test the emptiness of the resulting automaton.
			\end{itemize}
	\end{itemize}
\end{frame}

\begin{frame}{Checking Emptiness}
	\begin{itemize}
	\item Checking nonemptiness of $L(B)$ is equivalent to finding a strongly connected component that is reachable from an initial state and contains an accepting state.
	\item That is, the language $L(B)$ is nonempty iff \textcolor{blue}{there is a reachable accepting state with a cycle back to itself}.
	\item If we find such a cycle when checking the emptiness of $L(A) \cap L(B_{\neg f})$ , bad news, $A \not\models f$.
	\item Otherwise, good news, $A \models f$.
	\end{itemize}
\end{frame}

\begin{frame}[label=DFS3Figures]{Double DFS Algorithm}
\only<1>{\tikzset{
->, % makes the edges directed
>=stealth, % makes the arrow heads bold
every edge/.style={draw, thick, black},
node distance=2.7cm, % specifies the minimum distance between two nodes. Change if necessary.
every state/.style={thick, fill=gray!10}, % sets the properties for each ’state’ node
initial text=$ $, % sets the text that appears on the start arrow
}}

\only<2-3>{\tikzset{
->, % makes the edges directed
>=stealth, % makes the arrow heads bold
every edge/.style={draw, thick, cyan},
node distance=2.7cm, % specifies the minimum distance between two nodes. Change if necessary.
every state/.style={thick, fill=gray!10}, % sets the properties for each ’state’ node
initial text=$ $, % sets the text that appears on the start arrow
}}

	\begin{figure}[ht] % ’ht’ tells LaTeX to place the figure ’here’ or at the top of the page
	\centering % centers the figure
		\begin{tikzpicture}
			\node[state, initial] (1) {};
			\node[state, above right of=1] (2) {};
			\node[state, right of=2] (3) {};
			\node[state, right of=1] (4) {};
			\node[state, right of=4] (5) {};
			\node[state, below right of=1] (6) {};
			\node[state, right of=6] (7) {};
			\node[state, accepting, right of=7] (8) {};
			\node[state, below right of=7] (9) {};
			\node[state, right of=9] (10) {};
			\draw (1) edge node{} (2)
						(2) edge node{} (3)
						(1) edge node{} (4)
						(4) edge node{} (5)
						(1) edge node{} (6)
						(6) edge node{} (7)
						(7) edge node{} (8)
						(8) edge node{} (10)
						(10) edge node{} (9)
						(9) edge[black] node{} (7);
						\only<3>{
						\draw (8) edge[red] node{} (10)
						(10) edge[red] node{} (9)
						(9) edge[red] node{} (7);}
		\end{tikzpicture}
		\only<2>{\caption{DFS Tree}}
		\only<3>{\caption{When returning to an accepting state, it starts the second DFS (post-order)}}
	\end{figure}
\end{frame}

\begin{frame}{SPIN's structure}
\fontsize{6pt}{7.2}\selectfont
	\begin{figure}[ht]
	\centering
	\begin{tikzpicture}
		\tikzset{every node/.style={draw, rectangle, align=center, minimum height=0.8cm, minimum width=1.6cm, inner sep=2pt},
		arrow/.style = {thick,>=stealth}}

			\node(n1) {XSPIN\\Front-End\\(Tcl/Tk Code)};
			\node[below = of n1](n2) {PROMELA\\Parser};
			\node[right = of n2](n3) {LTL Parser\\and Translator};
			\node[below = of n2](n4) {2.\\Interactive\\Simulation};
			\node[left = of n4](n5) {1.\\Syntax Error\\Reports};
			\node[right = of n4](n6) {3.\\Verifier\\Generator};
			\node[below = of n6](n7) {Optimized\\Model Checker\\(ANSI C Code)};
			\node[below = of n7](n8) {Excutable\\On-The-Fly\\Verifier};
			\node[draw = none, left = 0cm of n8](n9) {Counter-\\Examples};
		
			\draw[arrow,<->] (n1) -- (n2);
			\draw[arrow,<->] (n2) -- (n3);
			\draw[arrow,->] (n2) -- (n4);
			\draw[arrow,->] (n2) -- (n5);
			\draw[arrow,->] (n2) -- (n6);
			\draw[arrow,->] (n6) -- (n7);
			\draw[arrow,->] (n7) -- (n8);
			\draw[arrow,->] (n8) -| (n4);
		\end{tikzpicture}
		%\caption{The structure of SPIN simulation and verification. Source: redrawn from [The SPIN Model Checker Figure 11.1 Gerard J. Holzmann]}
		\end{figure}
\end{frame}

\begin{frame}{If I want to get the code...}
	\begin{itemize}
		\item Download SPIN from the github and set it up.
		\item Suppose there is a system described in PROMELA named \code{example.pml}.
		\item \code{\$ spin -a example.pml   \hspace{1cm}\#generate c-verifier}
		\item \code{\$ cc -DNOREDUCE -o pan pan.c}
	\end{itemize}
\end{frame}

\begin{frame}[fragile]{While pan.c is unreadable...}

\begin{lstlisting}[basicstyle=\footnotesize\ttfamily]
/*
 * new_state() is the main DFS search routine in the verifier
 * it has a lot of code ifdef-ed together to support
 * different search modes, which makes it quite unreadable.
 * if you are studying the code, use the C preprocessor
 * to generate a specific version from the pan.c source,
 * e.g. by saying:
 * gcc -E -DNOREDUCE -DBITSTATE pan.c > ppan.c
 * and then study the resulting file, instead of this version
 */\end{lstlisting}
 	\begin{itemize}
 		\item \code{\$ gcc -E -DNOREDUCE -DBITSTATE pan.c > ppan.c}
 	\end{itemize}
\end{frame}


\begin{frame}[fragile]{The argument \code{-a}}
\begin{lstlisting}[basicstyle=\normalsize\ttfamily]
while (argc > 1 && argv[1][0] == '-'){ 
    switch (argv[1][1]) {
        case 'a': a_cycles = 1; break;
        /* ... */
    }
}\end{lstlisting}
 	\begin{itemize}
 		\item \code{-a} : find acceptance cycles.
 	\end{itemize}
\end{frame}  


\begin{frame}[fragile]{Trail}
\begin{lstlisting}[basicstyle=\normalsize\ttfamily]
trail = (Trail *) emalloc((maxdepth+3)*sizeof(Trail));
trail++; /* protect trpt-1 refs at depth 0 */
trpt = &trail[0];
\end{lstlisting}
 	\begin{itemize}
		\item Imagine that \code{trail} is a DFS stack.
 		\item \code{trpt} corresponds to the stack top.
 	\end{itemize}
\end{frame}

\begin{frame}[fragile]{Definition of Trail}
\begin{lstlisting}[basicstyle=\normalsize\ttfamily]
struct Trail {
  int   st;  /* current state */
  int   o_tt;
#ifdef PERMUTED
  uint  seed;
  uchar oII;
#endif
  uchar pr;  /* process id */
  uchar tau;  /* 8 bit-flags */
  uchar o_pm;  /* 8 more bit-flags */
}
\end{lstlisting}
\begin{itemize}
	\item There might be millions of states in the constructed automata.
	\item If we use \code{char}(1bytes) to keep boolean attributes of each state...
\end{itemize}
\end{frame}

\begin{frame}{Bitwise operation}
\begin{itemize}
	\item Every bit of \code{char} is a boolean variable by using bitwise operations \code{\&}(and), \code{|}(or), \code{$\sim$}(not).
	\item What do the following operations stand for?
	\begin{itemize}
		\item \code{a \& 4}\\	\only<2-4>{\textcolor{red}{value of the third bit is 0 or not}}
		\item \code{a |= 8}\\	\only<3-4>{\textcolor{red}{set the 4th bit to 1}}
		\item How to set the first and the sixth bit of \code{a} to 0?\\	\only<4>{\textcolor{red}{\code{a \&= $\sim$(1|32)}}}
	\end{itemize}
	\item save lots of memory used.
\end{itemize}
\end{frame}

\begin{frame}[fragile]{Meaning of bit-flags}
\begin{lstlisting}[basicstyle=\footnotesize\ttfamily]
#if 0
   Meaning of bit-flags on tau and o_pm:
   tau&1   -> timeout enabled
   tau&2   -> request to enable timeout 1 level up (in claim)
   tau&4   -> current transition is a  claim move
   tau&8   -> current transition is an atomic move
   tau&16  -> last move was truncated on stack
   tau&32  -> current transition is a preselected move
   tau&64  -> at least one next state is not on the stack
   tau&128 -> current transition is a stutter move
   o_pm&1  -> the current pid moved -- implements else
   o_pm&2  -> this is an acceptance state
   o_pm&4  -> this is a  progress state
   o_pm&8  -> fairness alg rule 1 undo mark
   o_pm&16  -> fairness alg rule 3 undo mark
   o_pm&32 -> fairness alg rule 2 undo mark
   o_pm&64 -> the current proc applied rule2
   o_pm&128 -> a fairness, dummy move - all procs blocked
#endif
\end{lstlisting}
\end{frame}

\begin{frame}[fragile]{settable()}
\begin{lstlisting}[basicstyle=\footnotesize\ttfamily]
void settable(void){ 
  Trans *T;
  Trans *settr(int, int, int, int, int, char *, int, int, int);
  trans = (Trans ***) emalloc(2*sizeof(Trans **));
  trans[0] = (Trans **) emalloc(13*sizeof(Trans *));

  trans[0][10] = settr(9,0,9,1,0,".(goto)", 0, 2, 0);
  T = trans[0][9] = settr(8,0,0,0,0,"DO", 0, 2, 0);
  T->nxt = settr(8,0,1,0,0,"DO", 0, 2, 0);
  trans[0][1] = settr(0,0,6,3,3,"a = ((a+1)\%3)", 0, 2, 0);
  /* ... */
}
\end{lstlisting}
\begin{itemize}
	\item Tool : LTL2BA
	\item 2 process, at most 13 states.
\end{itemize}
\end{frame}

\begin{frame}[fragile]{set accepting states}
\begin{lstlisting}[basicstyle=\normalsize\ttfamily]
accpstate[0] = (unsigned char *) emalloc(13);
accpstate[1] = (unsigned char *) emalloc(3);
accpstate[0][5] = 1;
/* ... */
do_the_search();
\end{lstlisting}
\begin{itemize}
	\item process 0的5號state是acceptance state
\end{itemize}
\end{frame}

\begin{frame}[fragile]{do\_the\_search()}
\begin{lstlisting}[basicstyle=\footnotesize\ttfamily]
void do_the_search(void){
  /* ... */
  for (i = 0; i < (int) now._nr_pr; i++){
    P0 *ptr = 
    (P0 *) (((unsigned char *)&now)+(int)proc_offset[i]);
    if (!(trpt->o_pm&2) && accpstate[ptr->_t][ptr->_p]){
      trpt->o_pm |= 2;
      break;
    }
  }  /* 檢查initial是不是也是acceptance state */
  /* ... */
  new_state();
}
\end{lstlisting}
\begin{lstlisting}[basicstyle=\footnotesize\ttfamily]
typedef struct P0 {
   unsigned _pid;  /* 0..255 */
   unsigned _t;    /* proctype (process id) */
   unsigned _p;    /* 目前處在的state */
} P0;
\end{lstlisting}
\end{frame}

\begin{frame}[fragile]{new\_state()}
\begin{itemize}
	\item 有段程式碼寫到
\end{itemize}
\begin{lstlisting}[basicstyle=\normalsize\ttfamily]
now._a_t |= (1|16|32);
/* 1 = 2nd DFS; (16|32) to improve hashing */
\end{lstlisting}
\begin{itemize}
	\item Although the first DFS and the second DFS are in the same funciton \textcolor{red}{\code{new\_state()}}, we can distinguish them easily by the value of \textcolor{red}{\code{now.\_a\_t \& 1}}.
	\item \code{new\_state()}分為許多階段,每個階段有自己的label,用\code{goto label\_name}在階段間轉移。
\end{itemize}
\end{frame}

\begin{frame}[fragile]{new\_state():Down(1st DFS)}
\begin{lstlisting}[basicstyle=\footnotesize\ttfamily]
Down:
  trpt->tau &= ~(16|32|64);
  if (depth >= maxdepth){
    if (!warned){ 
      warned = 1;
      printf("error: max search depth too small\n");
    }
    if (bounded){ 
      uerror("depth limit reached");
    }
    truncs++;
    (trpt-1)->tau |= 16;
    goto Up;
  }
\end{lstlisting}
\begin{itemize}
	\item Imagine that initial state is on the top of the automata.
\end{itemize}
\end{frame}

\begin{frame}[fragile]{new\_state():Allover(1st DFS)}
\begin{lstlisting}[basicstyle=\normalsize\ttfamily]
  /* ... */
for (t = trans[ot][tt]; t; t = t->nxt){
  (trpt+1)->o_t = t;
  if (!(_m = do_transit(t, II))){
    continue;
  }
  depth++; trpt++;
  /* ... */
\end{lstlisting}
\begin{itemize}
	\item Move to next state.
\end{itemize}
\end{frame}

\begin{frame}[fragile]{new\_state():Allover(1st DFS)}
\begin{lstlisting}[basicstyle=\footnotesize\ttfamily]
  if (a_cycles){
    int ii;
    for (ii = 0; ii < (int) now._nr_pr; ii++){
      if (accpstate[((P0 *)(((unsigned char *)&now)
      +(int)proc_offset[ii]))->_t]
      [((P0 *)(((unsigned char *)&now)
      +(int)proc_offset[ii]))->_p]){ 
        trpt->o_pm |= 2;
        break;
      }
    }
  }
   goto Down;
Up:
  /* ... */
}
\end{lstlisting}
\begin{itemize}
	\item Check if acceptance state and keep going.
\end{itemize}
\end{frame}

\begin{frame}[fragile]{new\_state():Allover(1st DFS)}
\begin{itemize}
	\item How about the program flow so far?
	\item
	\only <2-3>{\textcolor{red}{\code{Down}$\rightarrow$\code{Allover}$\rightarrow$\code{Down}$\rightarrow$\code{Allover}$\rightarrow\cdots\rightarrow$\code{Down}(Exceed \code{maxdepth})$\rightarrow$\code{Up}$\rightarrow\cdots$}}
	\item or\\
	\only<3>{\textcolor{red}{\code{Down}$\rightarrow$\code{Allover}$\rightarrow$\code{Down}$\rightarrow$\code{Allover}$\rightarrow\cdots\rightarrow$\code{Down}$\rightarrow$\code{Allover}(no next state)$\rightarrow$\code{Up}$\rightarrow\cdots$} }
\end{itemize}
\end{frame}

\begin{frame}[fragile]{new\_state():Up(1st DFS)}
\begin{lstlisting}[basicstyle=\normalsize\ttfamily]
Up:
  /* ... */
  goto R999;

R999:
  depth--; trpt--;
  /* ... */
\end{lstlisting}
\end{frame}

\begin{frame}[fragile]{new\_state():Done(1st DFS)}
\begin{lstlisting}[basicstyle=\normalsize\ttfamily]
Done:
  /* ... */
   if (a_cycles && (trpt->o_pm&2))
    checkcycles();
  /* ... */
 if (depth > 0){
   goto Up;
 }
\end{lstlisting}
\begin{itemize}
	\item o\_pm\&2  -> this is an acceptance state
	\item Since double DFS algorithm adopts post-order, it checks whether this state is an acceptance state or not after the DFS return. 
\end{itemize}
\end{frame}

\begin{frame}[fragile]{checkcycles()}
\begin{lstlisting}[basicstyle=\footnotesize\ttfamily]
void checkcycles(void){
  unsigned char o_a_t = now._a_t;
  unsigned char o_cnt = now._cnt[1];
  H_el *sv = trpt->ostate;
  now._a_t |= (1|16|32);
  now._cnt[1] = now._cnt[0];

  memcpy((char *)&A_Root, (char *)&now, vsize);
  A_depth = depthfound = depth; /* Acceptance Depth */

  new_state(); /* start 2nd DFS */

  now._a_t = o_a_t;
  now._cnt[1] = o_cnt;
  A_depth = 0; depthfound = -1;
  trpt->ostate = sv;
}
\end{lstlisting}
\begin{itemize}
	\item Save parameters and reset after 2nd DFS returns.
\end{itemize}
\end{frame}

\begin{frame}[fragile]{new\_state():Allover(2nd DFS)}
\begin{lstlisting}[basicstyle=\footnotesize\ttfamily]
if ((now._a_t&1) && depth > A_depth){
  int delta = 0 + 2;
  if (!memcmp((char *)&A_Root + delta,
    (char *)&now + delta, vsize - delta)){
    /* ... */
    depthfound = A_depth;
    uerror("acceptance cycle");
    goto Up;
  }
}
\end{lstlisting}
\begin{itemize}
	\item now.\_a\_t\&1 : in 2nd DFS.
	\item Check if the 2nd DFS reaches to the acceptance state which it starts form by comparing memory address. If so, \code{goto Up}. Otherwise, check if we reach the state which we have reached before(in 1st DFS).
\end{itemize}
\end{frame}

\begin{frame}[fragile]{new\_state():Allover(2nd DFS)}
\begin{lstlisting}[basicstyle=\footnotesize\ttfamily]
if (!JJ && (now._a_t&1) && depth > A_depth){
  int oj1 = j1_spin;
  unsigned char o_a_t = now._a_t;
  now._a_t &= ~(1|16|32);
  if (onstack_now()){  /*  check if we have reached in 1st DFS */
    II = 3;
  }
  now._a_t = o_a_t;
  j1_spin = oj1;
}
if (II == 3 && a_cycles && (now._a_t&1)){
  if (fairness && now._cnt[1] != 1){
    II = 0;
  } else {
same_case: if (Lstate) depthfound = Lstate->D;
    uerror("acceptance cycle");
    goto Up;
  }
}
\end{lstlisting}
\begin{itemize}
	\item now.\_a\_t\&1 : in the 2nd DFS.
	\item Check if the 2nd DFS reaches to the acceptance state which it starts form by comparing memory address. If so, \code{goto Up}. Otherwise, check if we reach the state which we have reached before(in 1st DFS).
\end{itemize}
\end{frame}

\begin{frame}[fragile]{new\_state():Up(2nd DFS)}
\begin{lstlisting}[basicstyle=\footnotesize\ttfamily]
Up:
  if ((now._a_t&1) && depth <= A_depth)
    return;
\end{lstlisting}
\begin{itemize}
	\item now.\_a\_t\&1 : in the 2nd DFS.
	\item A\_depth : Acceptance Depth.
	\item While returning to the starting acceptance state, return to \code{checkcycles()}
	\item Finally, we return to the \code{main} function and \code{wrapup()} those information we collected during the double DFS algorithm.
\end{itemize}
\end{frame}

\begin{frame}[fragile]{Program flow}
\begin{itemize}
	\item How about the whole program flow?
	\item 
	\textcolor{blue}{\code{Down}$\rightarrow$\code{Allover}$\rightarrow$\code{Down}$\rightarrow$\code{Allover}$\rightarrow\cdots\rightarrow$\code{Down}$\rightarrow$\code{Allover}(no next state)$\rightarrow$\code{Up}$\rightarrow$\code{Done}$\rightarrow$\code{Up}$\rightarrow$\code{Done}(This is an accepting state)$\rightarrow$}\code{checkcycles()}\textcolor{red}{$\rightarrow$\code{Down}$\rightarrow$\code{Allover}$\rightarrow$\code{Down}$\rightarrow$\code{Allover}$\rightarrow\cdots\rightarrow$\code{Down}$\rightarrow$\code{Allover}(this state has been reached in 1st DFS)$\rightarrow$\code{Up}$\rightarrow$\code{Done}$\rightarrow$\code{Up}(\code{depth <= A\_depth})$\rightarrow$return}
\end{itemize}
\end{frame}

\againframe{DFS3Figures}

\begin{frame}{Reference}
\begin{itemize}
	\item The SPIN Model Checker. Gerard J. Holzmann.
	\item LTL2BA : fast translation from LTL formulae to B¨uchi automata. Software written by Denis
Oddoux (v1.0) and modified by Paul Gastin (v1.2).\\
		\code{http://www.lsv.fr/∼gastin/ltl2ba/}
	\item SPIN website\\
		\code{http://spinroot.com/spin/whatispin.html
}
	\item Verification Options\\
		\code{http://spinroot.com/spin/Man/Pan.html
}
\end{itemize}
\end{frame}

\end{document}