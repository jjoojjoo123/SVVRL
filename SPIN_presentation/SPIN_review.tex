\documentclass[12pt]{beamer}
\usepackage{listings}
\usepackage{color}
\usepackage{xcolor}
%\usepackage{latexsym}
\usepackage{amsmath}
\usepackage[labelfont=bf]{caption}
\usepackage{graphicx}  %package graphic
\usepackage{siunitx}
\usepackage{tikz}
%\usepackage{algorithmicx}
%\usepackage[noend]{algpseudocode}
\usetikzlibrary{automata, positioning, arrows}
\usetheme{Boadilla}   
\usepackage{xeCJK}
%\usepackage{array}
%\usepackage{tabularx}
%\usepackage{mathtools}
\usepackage{listings}
%\usepackage{textcomp}
\usepackage[T1]{fontenc}
\usepackage{lmodern}
\usepackage{adjustbox}

\setCJKmainfont{微軟正黑體} 
\sisetup{
    group-separator={,},
    table-number-alignment=right
}


\setbeamerfont{title}{size=\Large,series=\bfseries}  % title size

\setbeamerfont{frametitle}{size=\large,series=\bfseries}  % frametitle size, also can size*=<pt>

\definecolor{themeColor}{rgb}{0.06, 0.3, 0.57}
\usecolortheme[named=themeColor]{structure}

\defbeamertemplate*{footline}{Dan P theme}
{
  \leavevmode%
  \hbox{%
  %\begin{beamercolorbox}[wd=.2\paperwidth,ht=2.25ex,dp=1ex,center]{author in head/foot}%
    %\usebeamerfont{author in head/foot}\insertshortauthor\expandafter\beamer@ifempty\expandafter{\beamer@shortinstitute}{}{~~(\insertshortinstitute)}
  %\end{beamercolorbox}%
  \begin{beamercolorbox}[wd=.7\paperwidth,ht=2.25ex,dp=1ex,center]{title in head/foot}%
    \usebeamerfont{title in head/foot}\insertshorttitle
  \end{beamercolorbox}%
  \begin{beamercolorbox}[wd=.3\paperwidth,ht=2.25ex,dp=1ex,right]{date in head/foot}%
    \usebeamerfont{date in head/foot}\insertshortdate{}\hspace*{2em}
\insertframenumber{} / \inserttotalframenumber\hspace*{2ex} 
  \end{beamercolorbox}}%
  \vskip0pt%
}
\makeatother

% Item include picture
\setbeamertemplate{itemize item}   % First Level item
{\includegraphics[height=0.33cm]{../Figures/golden-earth-on-white}}

\setbeamertemplate{itemize subitem} % Second level item
{\includegraphics[height=0.31cm]{../Figures/golden-sun-on-white}}

\setbeamertemplate{itemize subsubitem} % Third Level item
{\includegraphics[height=0.27cm]{../Figures/golden-paw-on-white}}

\definecolor{darkgold}{rgb}{0.765 0.64 0.0} % for highlighted text in black-and-white slides
\newcommand{\highlight}[1]{\structure{#1}}
\newcommand{\highlightb}[1]{\textcolor{blue}{#1}}
\newcommand{\highlightg}[1]{\textcolor{darkgold}{#1}}
\newcommand{\highlightr}[1]{\alert{#1}}
\newcommand{\code}[1]{\texttt{#1}}

%\newtheorem{problem}{Problem}
\mode<presentation>{\newtheorem{algorithm}{Algorithm}}
\mode<article>{\newenvironment{algorithm}{}{}}
%\newtheorem{solution}{Solution}

\newcommand{\hide}[1]{}
%\renewcommand{\highlightb}{\highlightg}

\newlength{\subtextwidth}
\setlength{\subtextwidth}{11cm}

\newenvironment{cbox}{
    \begin{center}
    \begin{tabular}{|l|}
    \hline
    \begin{minipage}[t]{\subtextwidth}}
    {\vspace{.25ex}
    \end{minipage}
    \hline
    \end{tabular}
    \end{center} }

\lstdefinestyle{mystyle}{
    basicstyle=\setmonofont{Consolas},
    columns=fullflexible,
    keepspaces=true,
    upquote=true,
    showstringspaces=false,
    commentstyle=\color{gray},
    keywordstyle=\color{teal},
    identifierstyle=\color{black},
    stringstyle=\color{brown},
    language=c
}

\lstset{style=mystyle,frame=single,}
\makeatletter
\lst@CCPutMacro
    \lst@ProcessOther {"2A}{%
    %   \lst@ttfamily 
         {\raisebox{1.5pt}{*}}% used with ttfamily
        %   \textasteriskcentered
          }% used with other fonts
    \@empty\z@\@empty
\makeatother

\newcommand{\mathdef}[1]{\relax\ifmmode #1\else $#1$\fi}
\newcommand{\true}{\mathdef{\mathit{true}}}
\newcommand{\false}{\mathdef{\mathit{false}}}
\renewcommand{\implies}{\mathdef{\rightarrow}}
\newcommand{\ifonlyif}{\mathdef{\leftrightarrow}}
\newcommand{\entails}{\mathdef{\vdash}}
\newcommand{\PROPS}{\mathdef{\mathit{PROPS}}}
\newcommand{\BOOL}{\mathdef{\mathit{BOOL}}}

\mode<presentation>{\title{\textsc{Spin}'s CheckEmpty Procedure\\ Code Tracing}}
\mode<article>{\title{Algorithms 2019: Analysis of Algorithms}}
%\subtitle{(Based on [Manber 1989])}
\author{林宏陽}
%\institute[IM.NTU]{Department of Information Management\\ National Taiwan University}
%\date[Algorithms 2019]{\null}
\mode<presentation>{\date[SVVRL]{\null}}
\mode<article>{\date{\today}}

\begin{document}
\begin{frame}
\maketitle
\end{frame}



\begin{frame}{\textsc{Spin}'s structure}
\fontsize{6pt}{7.2}\selectfont
    \begin{figure}[ht]
    \centering
    \begin{tikzpicture}
        \tikzset{every node/.style={draw, rectangle, align=center, minimum height=0.8cm, minimum width=1.6cm, inner sep=2pt},
        arrow/.style = {thick,>=stealth}}

            \node(n1) {X\textsc{Spin}\\Front-End\\(Tcl/Tk Code)};
            \node[below = of n1](n2) {PROMELA\\Parser};
            \node[right = of n2](n3) {LTL Parser\\and Translator};
            \node[below = of n2](n4) {2.\\Interactive\\Simulation};
            \node[left = of n4](n5) {1.\\Syntax Error\\Reports};
            \node[right = of n4](n6) {3.\\Verifier\\Generator};
            \node[below = of n6](n7) {Optimized\\Model Checker\\(ANSI C Code)};
            \node[below = of n7](n8) {Excutable\\On-The-Fly\\Verifier};
            \node[draw = none, left = 0cm of n8](n9) {Counter-\\Examples};
        
            \draw[arrow,<->] (n1) -- (n2);
            \draw[arrow,<->] (n2) -- (n3);
            \draw[arrow,->] (n2) -- (n4);
            \draw[arrow,->] (n2) -- (n5);
            \draw[arrow,->] (n2) -- (n6);
            \draw[arrow,->] (n6) -- (n7);
            \draw[arrow,->] (n7) -- (n8);
            \draw[arrow,->] (n8) -| (n4);
        \end{tikzpicture}
        %\caption{The structure of \textsc{Spin} simulation and verification. Source: redrawn from [The \textsc{Spin} Model Checker Figure 11.1 Gerard J. Holzmann]}
        \end{figure}
\end{frame}

\begin{frame}{\textsc{Spin}'s source files}
    \begin{itemize}
        \item \code{main.c} : main file
        \item \code{spinlex.c} : PROMELA parser
        \item \code{tl\_buchi.c, tl\_cache.c, tl\_lex.c, tl\_main.c, tl\_mem.c, tl\_parse.c, tl\_rewrt.c, tl\_trans.c} : LTL2BA
        \item from \code{pangen1.c} to \code{pangen7.c} : verifier code generator
        \item other files : other routines in \textsc{Spin}
    \end{itemize}
\end{frame}

\begin{frame}{Getting the code}
    \begin{itemize}
        \item Download \textsc{Spin} from the github and set it up.
        \item Suppose there is a system described in PROMELA named \code{example.pml}.
        \item \code{\$ spin -a peterson.pml   \hspace{1cm}\#generate c-verifier}
        \item \code{\$ cc -DNOREDUCE -o pan pan.c} \\ \code{\#compile the c-verifier to an excutable file}
        \item Option \code{-DNOREDUCE} : disables the partial order reduction algorithm.
        \item Use this option just for the easy code tracing purpose.
    \end{itemize}
\end{frame}

\begin{frame}[fragile]{While pan.c is unreadable...}

\begin{lstlisting}[basicstyle=\footnotesize]
/*
 * new_state() is the main DFS search routine in the verifier
 * it has a lot of code ifdef-ed together to support
 * different search modes, which makes it quite unreadable.
 * if you are studying the code, use the C preprocessor
 * to generate a specific version from the pan.c source,
 * e.g. by saying:
 * gcc -E -DNOREDUCE -DBITSTATE pan.c > ppan.c
 * and then study the resulting file, instead of this version
 */\end{lstlisting}
     \begin{itemize}
         \item \code{\$ gcc -E -DNOREDUCE -DBITSTATE pan.c > ppan.c}
     \end{itemize}
\end{frame}

\begin{frame}{Other files}
    \begin{itemize}
        \item \code{pan.t} : \textcolor{red}{t}ransition table and state table optimization
        \item \code{pan.m} : forward \textcolor{red}{m}ove
        \item \code{pan.b} : \textcolor{red}{b}ackward move
        \item \code{pan.h} : header file
        \item \code{pan.p} : BFS algorithm
    \end{itemize}
\end{frame}

\begin{frame}{Draw automata}
    \begin{itemize}
        \item \code{./pan -D > digraph}
        \item \code{python3 render.py digraph}
    \end{itemize}
\end{frame}

\begin{frame}[fragile]{Pseudo code (main)}
\begin{lstlisting}[basicstyle=\normalsize]
int main()
{
    8334        initialize trail
    8339        run()
}
\end{lstlisting}
\end{frame}

\begin{frame}[fragile]{Definition of Trail}
\begin{lstlisting}[basicstyle=\normalsize]
struct Trail {
  int   st;  /* current state */
  int   o_tt;
  
  uchar pr;  /* process id */
  uchar tau;  /* 8 bit-flags */
  uchar o_pm;  /* 8 more bit-flags */
  
  /* ... */
}
\end{lstlisting}
\end{frame}

\begin{frame}[fragile]{Meaning of bit-flags}
\begin{lstlisting}[basicstyle=\footnotesize]
#if 0
   Meaning of bit-flags on tau and o_pm:
   tau&1   -> timeout enabled
   tau&2   -> request to enable timeout 1 level up (in claim)
   tau&4   -> current transition is a  claim move
   tau&8   -> current transition is an atomic move
   tau&16  -> last move was truncated on stack
   tau&32  -> current transition is a preselected move
   tau&64  -> at least one next state is not on the stack
   tau&128 -> current transition is a stutter move
   o_pm&1  -> the current pid moved -- implements else
   o_pm&2  -> this is an acceptance state
   o_pm&4  -> this is a  progress state
   o_pm&8  -> fairness alg rule 1 undo mark
   o_pm&16  -> fairness alg rule 3 undo mark
   o_pm&32 -> fairness alg rule 2 undo mark
   o_pm&64 -> the current proc applied rule2
   o_pm&128 -> a fairness, dummy move - all procs blocked
#endif
\end{lstlisting}
\end{frame}

\begin{frame}[fragile]{Pseudo code (run)}
\begin{lstlisting}[basicstyle=\normalsize]
void run(void)
{
    5369        set transition table
    5421        Optimize state tables
    5452        initialize stack
    5466        add claim process  
        // allocate memory space and initialize _pid, initial state
    5474        add program processes   
    5485        do_the_search()
}
\end{lstlisting}
\end{frame}

\begin{frame}[fragile]{Set transition table}
\begin{lstlisting}[basicstyle=\scriptsize]
void settable(void)
{   Trans *T;
    Trans *settr(int, int, int, int, int, char *, int, int, int);
    trans = (Trans ***) emalloc(4*sizeof(Trans **));
    trans[2] = (Trans **) emalloc(9*sizeof(Trans *));
    trans[2][5] = settr(22,0,4,1,0,".(goto)", 0, 2, 0);    /* 2號process的5號state */
    T = trans[2][4] = settr(21,0,0,0,0,"DO", 0, 2, 0);
        T->nxt = settr(21,0,3,0,0,"DO", 0, 2, 0);
    T = trans[ 2][3] = settr(20,2,0,0,0,"ATOMIC", 1, 2, 0);
    T->nxt = settr(20,2,1,0,0,"ATOMIC", 1, 2, 0);
    trans[2][1] = settr(18,0,4,3,3,"(((P0._p==crit)&&(P1._p==crit)))", 1, 2, 0);
    reached2[2] = 1;
    trans[2][2] = settr(0,0,0,0,0,"assert(!(((P0._p==crit)&&(P1._p==crit))))",0,0,0);
    trans[2][6] = settr(23,0,7,1,0,"break", 0, 2, 0);
    trans[2][7] = settr(24,0,8,1,0,"(1)", 0, 2, 0);
    trans[2][8] = settr(25,0,0,4,4,"-end-", 0, 3500, 0);
    /* ... */
}
\end{lstlisting}
\end{frame}
    
\begin{frame}[fragile]{settr()}
\begin{lstlisting}[basicstyle=\footnotesize]
Trans *
settr( int t_id, int a, int b, int c, int d,
 char *t, int g, int tpe0, int tpe1)
{   Trans *tmp = (Trans *) emalloc(sizeof(Trans));    
    tmp->atom = a&(6|32);   /* if &2 = atomic trans; if &8 local */
    if (!g) tmp->atom |= 8;
    tmp->st = b;            /* the nextstate */
    tmp->tpe[0] = tpe0;     /* class of operation (for reduction) */
    tmp->tpe[1] = tpe1;     /* class of operation (for reduction) */
    tmp->tp = t;            /* src txt of statement */
    tmp->t_id = t_id;       /* transition id, unique within proc */
    tmp->forw = c;          /* index forward transition */
    tmp->back = d;          /* index return  transition */
    return tmp;
}
\end{lstlisting}
\end{frame}

\begin{frame}[fragile]{Optimize state tables}
\begin{lstlisting}[basicstyle=\footnotesize]
    retrans(0, 10, 1, src_ln0, reached0, loopstate0);
    /*  The Spin Model Checker p.538
       The state tables are optimized in three seperate steps by the verifier
       before the verification process begins. The original and intermediate versions
       of the tables can be generated by using the -d argument two, three, or four
       times in a row (e.g., by typing the command pan -d -d -d) */
    retrans(1, 10, 1, src_ln1, reached1, loopstate1);
    retrans(2, 9, 4, src_ln2, reached2, loopstate2);
\end{lstlisting}
\end{frame}

\begin{frame}[fragile]{Pseudo code (do$\_$the$\_$search)}
\begin{lstlisting}[basicstyle=\normalsize]
void do_the_search(void)
{
    6353        set current transition to claim move 
                /* the claim moves first */
    6391        new_state()
}
\end{lstlisting}
\end{frame}

\begin{frame}[fragile]{Pseudo code (newstate 1)}
\begin{lstlisting}[basicstyle=\scriptsize]
void new_state(void)
{
    6730        Down:
    6734            if depth >= maxdepth
    6754                goto Up
    6756        AllOver:
    6766            if in 2nd DFS
    6768                if(memory block matches 2nd DFS root)
                          {
    6780                    depthfound := A_depth
    6794                    goto Up
                          }
    6830            if in 2nd DFS
    6834                if state match on 1st dfs stack
                          {
    6859        same_case:  depthfound := Lstate->D;
    6863                    printf("acceptance cycle\n")
    6870                    goto Up
                          }
    6900            stack.push()
\end{lstlisting}
\end{frame}

\begin{frame}[fragile]{Pseudo code (newstate 2)}
\begin{lstlisting}[basicstyle=\tiny]
    6920        Stutter:
    6921            if current transition is a claim move
    6935                goto Veri0
    6969            for all processes p
                      {
    6971        Veri0:
    7006                for all transitions t
                          {
    7015                    if not do_transit(t, p) /* do_transit(t, p) return 0 if without executable clauses */
    7016                        continue
    7023                    depth++; trpt++
    7028                    moved down              // set trpt attributes
    7037                    for all processes including claim q
    7038                        if q is in accepting state
    7039                            set current state to acceptance state
    7050                    if t is an atomic move
                              {
    7051                        set current transition to atomic move
    7054                        if last transition is a claim move
    7055                            set current transition to claim move
    7056                        else
    7057                            set current transition to program move
                              }
    7058                    else
                              {
    7059                        if last transition is a claim move
    7060                            set current transition to program move
    7061                        else
    7062                            set current transition to claim move
                              }
    7078                    goto Down   /* pseudo-recursion */ 
\end{lstlisting}
\end{frame}

\begin{frame}[fragile]{Pseudo code (newstate 3)}
\begin{lstlisting}[basicstyle=\tiny]
    7079        Up:
    7081                    if in 2nd DFS && depth <= A_depth
    7082                        return
    7101~7189               undo move       //set last modified variable to oval (old value)
    7101~7189               goto R999
    7191        R999:
    7193                    depth--; trpt--
    7204                  }
    7229                if in claim automaton
    7230                    break
    7234              }
    7332        Done:
    7333            if current transition is not an atomic move
    7352                stack.pop()
    7359,7364     if current transition is a claim move && in 1st DFS
    7378                if this is an acceptance state
    7379                    checkcycles()   // prepare for 2nd DFS
    7383            if depth > 0
    7390                goto Up 
}

void checkcycles(void)
{
    8960~8973   store old state
    8974        A_depth := depthfound := depth
    8978        new_state()     /* start 2nd DFS */
    8980~8990   recover old state
}
\end{lstlisting}
\end{frame}

\begin{frame}[fragile]{do$\_$transit}
\begin{lstlisting}[basicstyle=\scriptsize]
unsigned char
do_transit(Trans *t, short II)
{ unsigned char _m = 0;
int tt = (int) ((P0 *)_this)->_p;
switch (t->forw) {
default: Uerror("bad forward move");
case 0:  /* if without executable clauses */
    return 0;
case 1: /* generic 'goto' or 'skip' */
    _m = 3; goto P999; /* ... */
/* PROC P1 */
case 5: // STATE 1 - peterson.pml:27 - [flag1 = 1] (0:0:1 - 1)
    reached[1][1] = 1;
    (trpt+1)->bup.oval = ((int)now.flag1);
    now.flag1 = 1;
    _m = 3; goto P999;
case 6: // STATE 2 - peterson.pml:28 - [turn = 1] (0:0:1 - 1)
    reached[1][2] = 1;
    (trpt+1)->bup.oval = ((int)now.turn);
    now.turn = 1;
    _m = 3; goto P999; /* ... */
P999: return _m;
}
\end{lstlisting}
\end{frame}


\begin{frame}{Reference}
\begin{itemize}
    \item The \textsc{Spin} Model Checker. Gerard J. Holzmann.
    \item LTL2BA : fast translation from LTL formulae to Büchi automata. Software written by Denis
Oddoux (v1.0) and modified by Paul Gastin (v1.2).\\
        \code{http://www.lsv.fr/∼gastin/ltl2ba/}
    \item \textsc{Spin} website\\
        \code{http://spinroot.com/spin/whatispin.html
}
\end{itemize}
\end{frame}

\end{document}